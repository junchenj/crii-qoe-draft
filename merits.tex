%!TEX root = main.tex
\section{Intellectual Merits and Broader Impacts}

%- Intellectual merit: A fundamental view of QoE optimization: taking an end-to-end view. potentially changing the way other internet services should be operated.
\mypara{Intellectual merits}
The proposed research will address interdisciplinary challenges and take an integrative approach to design and implement novel QoE-driven web services along three thrusts: 
(1) Quantifying the potentials of QoE-driven web backend to improve web QoE and backend resource utilization in the wild;
(2) Developing new architectural components to make web backend aware of the QoE sensitivity; and
(3) Developing QoE-driven control logic for common tasks in web backend systems, including message scheduling and replica selection.
This project also has the potential to increase resource efficiency and energy consumption of large-scale web services.
It might bear great implications for how the federated Internet systems should operate, from optimizing performance in isolation towards having an end-to-end perspective.
%4. Exploring the possibilities of QoE-driven optimization negatively affecting fairness.
% The key insight is a fundamentally new approach to QoE optimization over the federated Internet architecture, which argues that individual component should not be treated in isolation; instead, they should be made aware of the opportunities naturally inherent in the fact that end-users are affected by multiple components. 
% This bears implications not only for web backend, but many other systems too, such as content delivery networks, cloud providers, ISPs, cellular carriers, etc.

%- Broader impact: good for industry. achieving better quality/cost tradeoffs. teaching web services, quality of experience
\mypara{Broader impacts}
This research
% carried out as part of this proposal 
will have key implications for academia, industry, and education.


\begin{packeditemize}


\item {\em Education impact:}
The PI will develop new course materials and integrate it into his course offerings in systems and networking. 
The PI teaches two courses: Cloud Computing, and Introduction to Computer Systems. The PI will use the opportunities to design course projects (\eg measuring impact of cloud performance and building scalable prototypes for messaging systems over open-source platforms) and to attract undergraduates of University of Chicago to participate in research, deeply integrating them into the entire research lifecycle.


\item {\em Research impact:}
The proposed research will result in broad academic.
The PI will publish the results from the proposed research at top tier conferences and journals spanning both systems (e.g., SOSP, OSDI) and networking (e.g., SIGCOMM, NSDI, CoNEXT, IMC).
The PI has strong track record of publishing at these venues.
% dissimination

% cloud

\item {\em Industry and societal impact:}
The proposed work has significant implications for several commercial service providers, including CDNs, application developers, and cloud-hosted web service providers, many of which the PI has prior collaborations with, including Microsoft, Conviva, and Google. 
This work will inform the industry evolution by web service providers to more QoE-driven approaches.
% add conviva, microsoft, google
By delivering dramatic improvement in Internet QoE and saving resources of web service infrastructure, the research will have broader social impact. 

%\item {\em Societal impact:}
% improve qoe increase the engagement of users. resoruce savings  

\end{packeditemize}
