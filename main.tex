\documentclass{proposalnsf}

\usepackage{array}
\usepackage{subfigure}
\usepackage{paralist}
\usepackage{enumitem}
\usepackage{listings}
\usepackage{color}
\usepackage{wrapfig}
\usepackage{stfloats}
\usepackage{comment}
\usepackage{palatino}
\usepackage{titlesec}

\usepackage{longtable}
\usepackage{latexsym}
\usepackage{amsmath, amsthm, amssymb}
\usepackage{amsfonts}
\usepackage[format=plain, font=small, labelfont=bf]{caption}
\usepackage{fancyhdr}
\usepackage[pdftex]{graphicx}
\usepackage[pdftex,
	    colorlinks=true, 	
	    pdfstartview=FitH,
	    linkcolor=blue,
	    citecolor=blue,
	    urlcolor=blue,
	    filecolor=black
	    ]{hyperref}
\usepackage{lscape}
\usepackage[T1]{fontenc}
\usepackage{floatrow}
\usepackage{pdfpages}
\usepackage{todonotes}
\usepackage{xspace}

 \titlespacing{\section}{0pt}{5pt}{3pt}
 \titlespacing{\subsection}{0pt}{3pt}{2pt}

%a few commands to highlight issues in the proposal (\TODO, \CHECK, \DummyText)
\RequirePackage{color}
\definecolor{RED}{rgb}{1,0,0}
\definecolor{BLUE}{rgb}{0,0,1}
\definecolor{White}{rgb}{1,1,1}

\providecommand{\CHECK}[1]{{\protect\color{blue} #1 (check) }}
\providecommand{\DummyText}[1]{{\protect\color{white} #1}}

\providecommand{\andrew}[1]{{\color{blue} Andrew: #1}}
\providecommand{\anote}[1]{{\color{red} Todo: #1}}

\newlength{\bibitemsep}\setlength{\bibitemsep}{.2\baselineskip plus .05\baselineskip minus .05\baselineskip}
\newlength{\bibparskip}\setlength{\bibparskip}{0pt}
\let\oldthebibliography\thebibliography
\renewcommand\thebibliography[1]{%
  \oldthebibliography{#1}%
  \setlength{\parskip}{\bibitemsep}%
  \setlength{\itemsep}{\bibparskip}%
}

\newcommand{\ignore}[1]{}

\newcommand{\degrees}{$\!\!$\char23$\!$}

\renewcommand{\refname}{\centerline{References cited}}

% this handles hanging indents for publications
\def\rrr#1\\{\par
\medskip\hbox{\vbox{\parindent=2em\hsize=6.12in
\hangindent=4em\hangafter=1#1}}}

\def\baselinestretch{1}

\setlength{\columnwidth}{\textwidth}

\usepackage{floatrow}
\floatsetup[table]{capposition=top}

\newcolumntype{L}[1]{>{\raggedright\let\newline\\\arraybackslash\hspace{0pt}}m{#1}}
\newcolumntype{C}[1]{>{\centering\let\newline\\\arraybackslash\hspace{0pt}}m{#1}}
\newcolumntype{R}[1]{>{\raggedleft\let\newline\\\arraybackslash\hspace{0pt}}m{#1}}

\usepackage{wrapfig}
\newcommand{\jc}[1]{{\footnotesize\color{blue}{[#1]}}}

\newcommand{\fillme}{{\bf XXX}~}
\newcommand{\eg}{{\it e.g.,}\xspace}
\newcommand{\ie}{{\it i.e.,}\xspace}
\newcommand{\etal}{{\it et.~al}\xspace}
\newcommand{\bigO}{\mathrm{O}}

\newcommand{\mypara}[1]{\smallskip\noindent{\bf {#1}:}~}
\newcommand{\myparatight}[1]{\vspace{0.03cm}\noindent{\bf {#1}:}~}
\newcommand{\myparaq}[1]{\smallskip\noindent{\bf {#1}?}~}


\newcounter{packednmbr}


\newenvironment{packedenumerate}{\begin{list}{\thepackednmbr.}{\usecounter{packednmbr}\setlength{\itemsep}{0.5pt}\addtolength{\labelwidth}{-4pt}\setlength{\leftmargin}{\labelwidth}\setlength{\listparindent}{\parindent}\setlength{\parsep}{1pt}\setlength{\topsep}{0pt}}}{\end{list}}


\newenvironment{packeditemize}{\begin{list}{$\bullet$}{\setlength{\itemsep}{0.5pt}\addtolength{\labelwidth}{-4pt}\setlength{\leftmargin}{\labelwidth}\setlength{\listparindent}{\parindent}\setlength{\parsep}{1pt}\setlength{\topsep}{0pt}}}{\end{list}}


\newenvironment{packedpackeditemize}{\begin{list}{$\bullet$}{\setlength{\itemsep}{0.5pt}\addtolength{\labelwidth}{-4pt}\setlength{\leftmargin}{\labelwidth}\setlength{\listparindent}{\parindent}\setlength{\parsep}{1pt}\setlength{\topsep}{0pt}}}{\end{list}}


\newenvironment{packedtrivlist}{\begin{list}{\setlength{\itemsep}{0.2pt}\addtolength{\labelwidth}{-4pt}\setlength{\leftmargin}{\labelwidth}\setlength{\listparindent}{\parindent}\setlength{\parsep}{1pt}\setlength{\topsep}{0pt}}}{\end{list}}


\newcommand{\tightcaption}[1]{\vspace{-0.12cm}\caption{{\em #1}}\vspace{-0.13cm}}
%\newcommand{\tightsection}[1]{\vspace{-0.02in}\section{#1}\vspace{-0.03cm}}
%\newcommand{\tightsubsection}[1]{\vspace{-0.02in}\subsection{#1}\vspace{-0.03cm}}
\newcommand{\tightsection}[1]{\vspace{-0.08cm}\section{#1}\vspace{-0.08cm}}
\newcommand{\tightsubsection}[1]{\vspace{-0.08cm}\subsection{#1}\vspace{-0.08cm}}
%\newcommand{\tightsection}[1]{\vspace{-0.0cm}\section{#1}\vspace{-0.0cm}}
%\newcommand{\tightsubsection}[1]{\vspace{-0.0cm}\subsection{#1}\vspace{-0.0cm}}
\newcommand{\tightsubsubsection}[1]{\vspace{-0.01in}\subsubsection{#1}\vspace{-0.01cm}}


\newcommand{\taskref}[1]{\textbf{Thrust~}\ref{#1}}
\newcommand{\subtaskref}[1]{\textbf{Task~}\ref{#1}}
\newcounter{tasknmbr}
\newcounter{tasklabel}

\newcounter{subtasknmbr}[tasknmbr]
\newcounter{subtasklabel}[tasklabel]

\renewcommand{\thesubtasklabel}{\textbf{\thetasknmbr.\thesubtasknmbr}}

\renewcommand{\thetasklabel}{\textbf{\thetasknmbr}}
\newenvironment{task}{
\begin{list}{\textbf{Task }\thetasklabel:~}{\usecounter{tasklabel}\stepcounter{tasknmbr}\setlength{\labelwidth}{0pt}\setlength{\labelsep}{0pt}\setlength{\leftmargin}{0in}\noindent \rule{\textwidth}{1pt}\vspace{-9pt} \rule{\textwidth}{1pt}\vspace{-9pt}\item \bf\em}}{\\[-7pt]\end{list}\vspace{-11pt}\noindent\rule{\textwidth}{1pt}\vspace{-9pt} \rule{\textwidth}{1pt}}

\newenvironment{subtask}{\vspace{-0.05cm}
\begin{list}{ \textbf{Task }\thesubtasklabel:~}{\usecounter{subtasklabel}\stepcounter{subtasknmbr}\setlength{\labelwidth}{0pt}\setlength{\labelsep}{0pt}\setlength{\leftmargin}{0in}\noindent \rule{\textwidth}{1pt}\vspace{-9pt}\item\em}}{\\[-7pt]\end{list} \vspace{-9pt}\noindent\rule{\textwidth}{1pt}}

%%%%%%%%%%%%%%%%%%%%%%%%%%%%%%%%%%%%%%%
%%%%% Document starts here

\begin{document}

%%%%%%%%%%%%%%%%%%%%%%%%%%%%%%%%%%%%%%%
% A - COVER SHEET: Produced by fastlane, type in information there.

%%%%%%%%%%%%%%%%%%%%%%%%%%%%%%%%%%%%%%%
% B - PROJECT SUMMARY
%{\bf \title} \\*[3mm]

%% {\bf Overview:} Each proposal must contain a summary of the proposed project not more than {\bf one page in length}. The Project
%% Summary consists of an overview, a statement on the intellectual merit of the proposed activity, and a statement
%% on the broader impacts of the proposed activity.

%% The overview includes a description of the activity that would result if the proposal were funded and a statement
%% of objectives and methods to be employed.  

%% The Project Summary should be written in the third person, informative to other persons working in
%% the same or related fields, and, insofar as possible, understandable to a scientifically or technically 
%% literate lay reader. It should not be an abstract of the proposal.
%% If the Project Summary contains special characters it may be uploaded as a Supplementary Document.
%% {\bf Project Summaries submitted as a PDF must be formatted with separate headings for the overview, statement on the
%% intellectual merit of the proposed activity, and statement on the broader impacts of the proposed activity}. Failure
%% to include these headings may result in the proposal being returned without review.
%% Additional instructions for preparation of the Project Summary are available in FastLane.\\
% \begin{center}
% %{\bf\large PROJECT DESCRIPTION} \\
% %\smallskip
% \vskip -1em
% {\sc  \small CRII: NeTS: Towards QoE-Aware Web Services}
% \end{center}

\begin{center}
{\bf\Large CRII: NeTS: \\Unleashing the Potential of QoE-Driven Web Backend} \\
\smallskip
\end{center}

\renewcommand{\thepage} {B--\arabic{page}}
%\newpage
%%!TEX root = main.tex

\vskip -.75em
\mypara{Summary}


%%%%%%%%%%%%%%%%%%%%%%%%%%%%%%%%%%%%%%%
% C - TABLE OF CONTENTS: Automatically generated by fastlane.


%%%%%%%%%%%%%%%%%%%%%%%%%%%%%%%%%%%%%%%
% D - PROJECT DESCRIPTION

% reset page numbering to 1.  This is helpful, since the text can only
% be 15 pages (unless otherwise specified, see individual solicitations), 
% and reviewers will want to believe we've kept within those limits

%%  \newpage
%\tableofcontents
%\newpage

\pagenumbering{arabic}
\renewcommand{\thepage} {D--\arabic{page}}


%% The Project Description should provide a clear statement of the work to be undertaken and must include:
%% objectives for the period of the proposed work and expected significance; relation to longer-term goals of the PI's
%% project; and relation to the present state of knowledge in the field, to work in progress by the PI under other
%% support and to work in progress elsewhere.

%% The Project Description should outline the general plan of work, including the broad design of activities to be
%% undertaken, and, where appropriate, provide a clear description of experimental methods and procedures.
%% Proposers should address what they want to do, why they want to do it, how they plan to do it, how they will
%% know if they succeed, and what benefits could accrue if the project is successful. The project activities may be
%% based on previously established and/or innovative methods and approaches, but in either case must be well
%% justified. These issues apply to both the technical aspects of the proposal and the way in which the project may
%% make broader contributions.

%% The Project Description must contain, as a separate section within the narrative, a section labeled ``Broader
%% Impacts of the Proposed Work''. This section should provide a discussion of the broader impacts of the proposed
%% activities. Broader impacts may be accomplished through the research itself, through the activities that are
%% directly related to specific research projects, or through activities that are supported by, but are complementary to 
%% the project. NSF values the advancement of scientific knowledge and activities that contribute to the
%% achievement of societally relevant outcomes. Such outcomes include, but are not limited to: full
%% participation of women, persons with disabilities, and underrepresented minorities in science, technology, engineering, and
%% mathematics (STEM); improved STEM education and educator development at any level; increased public
%% scientific literacy and public engagement with science and technology; improved well-being of individuals in
%% society; development of a diverse,globally competitive STEM workforce; increased partnerships between
%% academia, industry, and others; improved national security; increased economic competitiveness of the United
%% States; and enhanced infrastructure for research and education.

%% Plans for data management and sharing of the products of research, including preservation, documentation, and
%% sharing of data, samples, physical collections, curriculum materials and other related research and education
%% products should be described in the Special Information and Supplementary Documentation section of the
%% proposal (see GPG Chapter II.C.2.j. for additional instructions for preparation of this section).

%\vspace{0.2in}

%\tableofcontents

%\newpage
%!TEX root = main.tex

\section{Introduction}

\noindent A key challenge facing Internet services today is how to share resources across users to maximize the user-perceived {\em QoE} (Quality of Experience) by minimizing their end-to-end delay.
Maintaining desirable level of user-perceived QoE is critical, with reducing a few hundred milliseconds from the page load time means millions of dollars.
%The fundamental challenge facing large-scale web service providers and (\eg Microsoft, Facebook, Akamai, Google, Amazon) is how the backend systems should share its resources in order to optimize user-perceived {\em QoE} (Quality of Experience). 
%Maintaining desirable level of user-perceived QoE critical for their revenue models.
A delay penalty of 400ms in Google search responses reduces search volume by 0.74\%; and 500ms of latency for Bing decreases revenue by 1.2\%~\cite{google-revenue,bing-revenue}; For Amazon, an additional latency of 100ms means a 1\% drop in sales~\cite{amazon-revenue}.
Yet, despite substantial efforts (\eg~\cite{shandian,gaze,rosen2017push,jalaparti2013speeding}), maintaining desirable QoE remains a challenge, with average page loading time of Facebook requests being over 3000ms~\cite{mystery}.
% with 50\% users of some popular website spending over 30\% of page loading time on the web backend~\cite{mystery}.

We argue that the key missing piece of today's Internet architecture is that individual Internet systems, \eg web backends, CDNs, and ISPs, are agnostic to their impact on individual users' real-time QoE, especially the difference in their impact across users.
As a consequence, it is difficult for these systems to direct optimizes QoE. 
Instead, our overarching thesis that these systems should be driven directly by the real-time information about their impact on the QoE across users, which could substantially improve QoE as well as resource efficiency, without adding new resources.
While there have been similar efforts in the past (\eg~\cite{alto,frank2013pushing,xie2008p4p,jiang2009cooperative}) toward similar end-to-end quality optimization, our project is inspired by several favorable recent trends, including ``use-case pulls'', \eg the prevalence of QoE-driven revenue models (\eg~\cite{akamai-report,dobrian2011understanding}), as well as ``technological pushes'', \eg the wide use of tracing from the clients and the backend systems (\eg~\cite{mystery,zhao2014lprof}). 

This proposal explores the benefits and challenges of our thesis in the context of how web backend systems should optimize web QoE by leveraging user heterogeneity. 

\mypara{Limitation of today's web backend}
%The web backend systems today have no direct visibility of the QoE of each web request when it arrives, 
Because the end-to-end delay (page loading time) of a web request is affected by the delay of many non-backend systems (ISPs, client-side software, etc) which is beyond the scope of the web service providers, the web backend systems have no direct visibility of how much impact it has on the QoE of individual requests.
As a result, the web backend systems seek to minimize the overall {\em backend delay} (\eg the mean, tail values, or the probability of missing some SLA deadline), under the assumption that backend delay of $n$~ms has the {\em same} impact on any request.\footnote{Modulo the content-specific (\eg web page type) or user-specific (\eg free vs. premium subscription) factors.}
%Page load time, which we refer to as {\em end-to-end delay}, generally consists of three parts: client-side delay, wide-area network (WANs) delay, and backend delay.
%- Web services, like applications running in the cloud, have been basing their optimizations on the goal of improving server-side latency (sometimes the fraction of users meeting some fixed deadline)
%Because of the federated nature of Internet architecture, web service providers do not have full control over all types of delays.
%---to them, WANs and clients devices are largely blackboxes operated, not by the web services, but by ISPs, cellular carriers, and device vendors.
%(while web browsers and apps are developed by the web service providers, the client-side performance is largely decided by how OS share resources among multiple applications).
%With web service providers only controlling the web backend, the performance metric they focus on optimizing is the 
%%Thus, instead of optimizing for QoE directly, today's web services focus on reducing the 
%%different requests have the same {\em QoE sensitivity to backend delay}
%{\em backend delay}, under the assumption that a backend delay of $n$~ms has the {\em same} impact on any request.\footnote{Modulo the content-specific (\eg web page type) or user-specific (\eg free vs. premium subscription) factors.}
% That is, a backend delay of $n$~ms has the same effect on the QoE of any request.
%For instance, they minimize the mean/tail backend delay or the fraction of requests whose backend delay exceeds some deadline (\eg 300ms)~\cite{??,??}.
%- This project takes a step back and asks a different question: does the latency have the same impact on user QoE? 
% In doing so, 
%all requests are optimized with the same objective function of backend response time; 
% an implicit assumption is that different requests have the same {\em QoE sensitivity to backend delay} (modulo content-/user-specific factors, such as web page type or subscription type, etc);
% that is, a backend delay of $n$~ms has the same effect on the QoE of different requests.

We take a step back, and ask {\em ``does the backend delay really have the same impact the QoE of any web request?''}
%- The answer is no, which has profound impact on how web services should be built. [Give a simple example here.] In essence, this means giving each ``priority'', in terms of resources and scheduling, is cost-inefficient and suboptimal. [Give a simple example. resources wasted for users who are screwed already]
Our answer is {\em no}, and such heterogeneous sensitivity of QoE to backend delay is pervasive among requests of the same application. 
This is due to the fact that how much impact the backend delay has on a request depends on the non-backend delay experienced by a request (\eg ISP routing, client-side resource allocation by the device OS), which can have a great variability among requests and over time~\cite{timecard,dqbarge}.
%Two observations contribute to this conclusion: the non-linear relationship between page load time and QoE~\cite{??} and the fact that the WAN/client delay varies among requests~\cite{timecard,dqbarge}, {\em the QoE sensitivity to backend delay varies among requests.}
For instance, the QoE of a web request that has spent 50ms on wide-area networks tends to be more sensitive to 10ms backend delay than a request that has already spent 500ms on the network.
By falsely assuming requests are equally sensitive to the backend delay, traditional web backend system (Figure~\ref{fig:intro-overview}(a)) might waste resources on requests that are insensitive to the backend delay, and/or have suboptimal QoE (\eg using inadequate resources on requests whose QoE is critically dependent on the backend delay. 

\mypara{Our approach} 
This proposal introduces ``QoE sensitivity'' as a new dimension for web QoE optimization, and propose to develop {\bf QoE-driven web backend} systems (Figure~\ref{fig:intro-overview}(b)). 
We argue that embracing the heterogeneity in QoE's sensitivity to backend delay has profound implications for backend systems: 
it should allocate resources in a way that favors the requests whose QoE is more sensitive to backend delay to improve the overall QoE while having minimal drop in the QoE of other requests.
For instance, in a trace-driven simulation (\S\ref{sec:quantifying}), we found that a simple QoE-driven resource allocation policy can raise user experience (measured in user engagement) 50\% closer to optimal (\ie zero backend delay) than a QoE-agnostic baseline web backend.
%should allocate their limited resources. 
%By taking into account the QoE's sensitivity to the backend delay, 
%\jc{bring up some concrete improvement numbers}
%\jc{need to highlight that this is not because application differents}

\begin{figure}[t]
	\centering
	\vspace{-0.5cm}
%	\hspace{0.6cm}
	\includegraphics[width=0.75\textwidth]{figs/intro-overview-new.pdf}
	\vspace{-0.1cm}
	\tightcaption{We propose to re-architect (a) today's web backend which seeks to minimize the backend delays into (b) a QoE-driven web backend which seeks to minimize the impact of backend delay on QoE.}
	\label{fig:intro-overview}
\end{figure}

%\jc{give a figure to contrast optimization of backend in-isolation vs. QoE-aware.}

%- Research goal: This project proposes that the web service backend should be aware of the QoE sensitivity. This effectively changes how one formulates the web service optimization problem.


%The key difference is that 
%Unlike a traditional backend which seeks to minimize the backend delays, a QoE-driven backend seeks to minimize the impact of backend delays on user-perceived QoE.

Despite its promise, we face two key challenges.
First, when a request is received, the backend needs to estimate in real time the sensitivity of the request's QoE to the backend delay with sufficient accuracy, and propagate the information across individual components of the backend. This information is not readily available in today's backend.
Second, the resource allocation and scheduling logic of web backend subsystems need to be updated to leverage the heterogeneity across users. The challenge arises from the fact that requests need to be set with different priorities in real time depending on their non-backend delay.
There are other problem inherent to the QoE-driven approach, including control stability (\ie the backend system becomes sensitive to any changes in non-backend delay), and exacerbating QoE unfairness (\ie the backend system favors sensitive users who may already have better QoE than some others).
%Through developing novel algorithms and architectural components, we show that a {\bf QoE-driven web backend} (Figure~\ref{fig:intro-overview}(b)), which is aware of and embraces the differences of QoE sensitivity across requests, can substantially {\em improve the resource/QoE tradeoffs} of web backend; \ie better QoE without using more resources, or saving resources without degrading QoE. 
% Note that being QoE sensitivity does not require expensive infrastructure changes (\eg adding hardware or changing software stack).

\mypara{Research plan}
This proposal explores the problem space by focusing on addressing the first two challenges, and will also discuss other questions (\eg control instability and unfairness).
We divide the proposed research into three main tasks. 
%We use the following roadmap to thoroughly examine the benefits and challenges of QoE-sensitivity-aware web service backend.

\begin{packeditemize}
\item{\bf Quantifying potential benefits (Task \#1).}
We will use a mix of user studies and analysis of industry datasets to quantify the potential improvement (in terms of QoE and resource savings) brought by the QoE-driven web backend in real-world workloads and identify the opportunities of QoE-driven optimizations in existing web backend systems.%, and use measurement dataset from large-scale web services to understand its potential in the real-world traffic patterns.

\item{\bf Estimating per-request QoE in real time (Task \#2).}
We will develop new tracing infrastructure and QoE prediction models to estimate the impact of backend delay on requests' QoE in real time. We will investigate the possibilities of incremental deployment by reusing the existing tracing and telemetry infrastructure in today's web backend systems.

\item{\bf QoE-driven control algorithms (Task \#3).}
We will develop novel QoE-driven control policies for web backend, including resource allocation, scheduling, and replica selection. Our design goals are (1) that the policies should achieve near-optimal QoE with minimal decision-making overhead, and (2) that their implementation should be amenable to existing systems.

\end{packeditemize}


\mypara{PI qualification}
The PI's expertise includes computer networking, Internet QoE, and data analytics systems.
He has published 11 peer-reviewed research papers (6 first-authored) in top-tier networking and systems conferences (\ie SIGCOMM, NSDI, CoNEXT).
More importantly, the PI has a deep understanding of Internet QoE. His doctoral dissertation, titled ``Enabling Data-Driven Optimization of Quality of Experience in Internet Applications'', is among the first systematic applications of data-driven approach to improving Internet QoE, and some of the proposed QoE prediction techniques (\eg~\cite{cfa,c3}) have led to real-world deployment and impact. The dissertation won the CMU SCS Doctoral Dissertation Award and was nominated for ACM Dissertation Award.
%During his PhD and postdoctoral years, 
He has extensive collaborations with Microsoft, Conviva, and Google, which will help the proposed research gain insights from the industry and provide viable paths to deployment.
%These strong connection

This project will provide the material needed for further research on the QoE-driven systems and is part of the necessary and critical steps for the PI to achieve research independence.



%\newpage










\begin{comment}


\noindent The ecosystem of web applications critically depends on maintaining desirable user-perceived QoE (quality of experience).
% QoE depends critically on web page loading time, 
Yet, despite tremendous efforts,
%(\eg cutting tail latency via redundancy or pushing caches closer to end users), 
many users still suffer from suboptimal QoE.
%Unlike previous approaches such as cutting tails of response time or pushing caches closer to end users, 
One fundamental issue is that, due to the federated Internet architecture, it is difficult for the web backend to measure the impact of its delay on a web request's QoE in real time. As a result, today's web backends seek to minimize the delay of every request to the same level.

This proposal introduces a new dimension for optimizing web QoE, which has been ignored by today's web backend: embracing the {\em inherent heterogeneity} of how much impact the {\em backend delay} has on different users' QoE. Such heterogeneity results from not only different applications or services, but more importantly difference in the delay of non-backend systems (\eg ISPs).
%, even if they request the same type of application/service.
% {\bf QoE sensitivity to backend delay} across users. More importantly, such heterogeneity is prevalent even if the users request the same type of application/service.
%, \ie how sensitive a user's QoE is to the web backend delay.
For instance, a web request that has spent 50ms on wide-area networks tends to be more sensitive to 10ms delay of web backend than a request that has already spent 500ms on the network. 
Such discrepancy in QoE's sensitivity to backend delay has profound implications---web requests previously seen as indistinguishable by the backend can now be treated differently so as to improve overall QoE by favoring the requests whose QoE is more sensitive to backend delay without hurting other requests' QoE or adding any new resources.
% on how web backend should allocate its limited resources across requests. In essence, 
%We show early promising result that by making existing the web backend aware of QoE sensitivity, we could improve both QoE and resource efficiency than existing solutions.
To fully unleash the potential, we propose to re-architect today's web backend systems by elevating {\em the sensitivity of QoE to the backend delay} as a key factor in the control logic of web backend.
% investigate new opportunities to improve QoE and save resources by making web services aware of QoE sensitivity (\eg better scheduling policies and replica selection policies) and 
We face two key technical challenges: obtaining real-time QoE feedback as input to the backend systems, and QoE-driven algorithms that achieve better QoE/resource tradeoffs.
%balancing QoE and efficiency, estimating the QoE sensitivity, and addressing fairness issues.
We plan to quantify the potential of QoE-driven web backend in the wild, and present new system component to provide real-time QoE estimation, and novel QoE-driven scheduling and resource allocation policies for better QoE and resource efficiency.
% (1) We quantify its potential benefits in QoE and resource savings.
% (2) We propose novel algorithms for QoE-aware scheduling and resource allocation of web services. 
% (3) We present novel system designs and implement prototypes that make web services QoE-aware in practice.

A future generalization of the project is that today's ecosystems of applications (\eg web, video streaming) are such that it is difficult for individual systems (\eg CDNs, ISPs, cloud) to explicitly drive decision-making towards better QoE. 
Our overarching vision is that by driving these systems explicitly with user-perceived QoE, we can unleash new opportunities to significantly improve the resource-efficiency and QoE.
%not built to directly optimize user-perceived QoE, and we believe the key missing piece is to let every system (including CDNs, ISPs, web service backend) be aware of the impact of their control decisions on QoE. 
By enabling web backend systems to embrace heterogeneous QoE sensitivity, this proposal is a first step towards the vision. 


% Thus, the goal of each subsystems in a large web service, such as web server or key-value store, should be to optimize the overall QoE of many users given limited resources. 
% A common approach to achieving this goal is for each subsystem to optimize some ``local'' performance metrics measured within its scope (\eg server-side delay) over all users, and the intuition is that if each subsystem follows the approach, it will optimize the overall QoE of users. 
% We argue, however, that this approach only achieves suboptimal QoE and can use more resources than necessary. 
% Our key observation is that {\em the impact of a subsystem's performance on a user's perceived QoE varies greatly among users} (modulo web page type, business relationship), so when sharing resources across users, each subsystem should take into account its impact on each user's QoE.
% One typical sources 
% This has profound implication on how web services should be optimized, and opens up many several new opportunities.


\section{Introduction}

% - QoE is important and our goal is to improve QoE for Web Services.
\noindent The fundamental challenge facing large-scale web service providers (\eg Microsoft, Facebook, Google) is how the backend system should  share its resources across users to optimize user-perceived QoE (Quality of Experience). 
%Web QoE has been shown to be critically dependent on web page loading time~\cite{??,??}.
Despite substantial efforts~\cite{??,??,??} and more resources~\cite{??,??,??}, maintaining desirable QoE remains a challenge with 50\% users of some popular website spending over 30\% of page loading time on the web backend~\cite{mystery}.
%whose experience could have been improved from bad to good if the backend delay is zero~\cite{dqbarge}.
%Their business models, based on advertisement or subscription, are driven by user engagement, for which QoE is believed to play a vital role (among other factors such as content, user interfaces).

\mypara{Limitation of today's web backend}
We argue that a key reason is that the web backend systems do not have the direct visibility of the QoE of each web request when it arrives. 
This is because QoE is affected by the performance of many non-backend systems (ISPs, client-side software, etc), while the web service providers only control the web backend systems.
As a result, web service providers focus on minimizing the {\em backend delay}, with the assumption that backend delay of $n$~ms has the {\em same} impact on any request.\footnote{Modulo the content-specific (\eg web page type) or user-specific (\eg free vs. premium subscription) factors.}
%Page load time, which we refer to as {\em end-to-end delay}, generally consists of three parts: client-side delay, wide-area network (WANs) delay, and backend delay.
%- Web services, like applications running in the cloud, have been basing their optimizations on the goal of improving server-side latency (sometimes the fraction of users meeting some fixed deadline)
%Because of the federated nature of Internet architecture, web service providers do not have full control over all types of delays.
%---to them, WANs and clients devices are largely blackboxes operated, not by the web services, but by ISPs, cellular carriers, and device vendors.
%(while web browsers and apps are developed by the web service providers, the client-side performance is largely decided by how OS share resources among multiple applications).
%With web service providers only controlling the web backend, the performance metric they focus on optimizing is the 
%%Thus, instead of optimizing for QoE directly, today's web services focus on reducing the 
%%different requests have the same {\em QoE sensitivity to backend delay}
%{\em backend delay}, under the assumption that a backend delay of $n$~ms has the {\em same} impact on any request.\footnote{Modulo the content-specific (\eg web page type) or user-specific (\eg free vs. premium subscription) factors.}
% That is, a backend delay of $n$~ms has the same effect on the QoE of any request.
For instance, they minimize the mean/tail backend delay or the fraction of requests whose backend delay exceeds some deadline (\eg 300ms)~\cite{??,??}.
%- This project takes a step back and asks a different question: does the latency have the same impact on user QoE? 
% In doing so, 
%all requests are optimized with the same objective function of backend response time; 
% an implicit assumption is that different requests have the same {\em QoE sensitivity to backend delay} (modulo content-/user-specific factors, such as web page type or subscription type, etc);
% that is, a backend delay of $n$~ms has the same effect on the QoE of different requests.

We take a step back, and ask {\em ``does the backend delay really have the same impact the QoE of any web request?''}

%- The answer is no, which has profound impact on how web services should be built. [Give a simple example here.] In essence, this means giving each ``priority'', in terms of resources and scheduling, is cost-inefficient and suboptimal. [Give a simple example. resources wasted for users who are screwed already]
\mypara{Our insight} 
Our answer is {\em no}.  More importantly, even if the requests have no application-level differences (\eg web vs. video), such heterogeneous QoE sensitivity can still result from the differences in non-backend delay (\eg wide-area network routing, client-side software)~\cite{timecard,dqbarge}.
%Two observations contribute to this conclusion: the non-linear relationship between page load time and QoE~\cite{??} and the fact that the WAN/client delay varies among requests~\cite{timecard,dqbarge}, {\em the QoE sensitivity to backend delay varies among requests.}
For instance, the QoE of a web request that has spent 50ms on wide-area networks tends to be more sensitive to 10ms backend delay than a request that has already spent 500ms on the network. 

Realizing the heterogeneity in QoE's sensitivity to backend delay has profound implications for how backend systems should allocate their limited resources. 
By falsely assuming requests are equally sensitive to the backend delay, traditional web backend (Figure~\ref{fig:intro-overview}(a)) can waste precious resources (\eg wasting resources to optimize requests insensitive to the backend delay) and have suboptimal QoE (\eg using inadequate resources on requests critically dependent on the backend delay). 
Instead, we argue that the web backend system should take into account the QoE's sensitivity to the backend delay, and allocate resources in a way that improves overall QoE by favoring the requests whose QoE is more sensitive to backend delay without hurting other requests' QoE or adding any new resources.
In a trace-driven simulation (\S\ref{subsec:example}), we found that such a QoE-driven resource allocation policy can make user engagement 50\% closer to the optimal outcome (zero backend delay) than a QoE-agnostic baseline.
%\jc{bring up some concrete improvement numbers}
%\jc{need to highlight that this is not because application differents}

\begin{figure}[t]
	\centering
	\vspace{-0.5cm}
	\hspace{0.6cm}
	\includegraphics[width=0.75\textwidth]{figs/intro-overview.pdf}
	\vspace{-0.3cm}
	\caption{We propose to re-architect (a) today's web backend which seeks to minimize the backend delays into (b) a QoE-driven web backend which seeks to minimize the impact of backend delay on QoE.}
	\label{fig:intro-overview}
\end{figure}

%\jc{give a figure to contrast optimization of backend in-isolation vs. QoE-aware.}

%- Research goal: This project proposes that the web service backend should be aware of the QoE sensitivity. This effectively changes how one formulates the web service optimization problem.

%This proposal introduces ``QoE sensitivity'' as a new dimension for web backend QoE optimization, 
We propose to develop {\bf QoE-driven web backend} systems (Figure~\ref{fig:intro-overview}(b)). 
The key difference is that traditional backend seeks to minimize the backend delays, but a QoE-driven backend seeks to minimize the impact of backend delays on user-perceived QoE.

Despite its promise, QoE-driven backend faces two key challenges.
First, we need to estimate the real-time QoE information of each web request with sufficient accuracy when the request reaches the backend. 
Second, we need to enable QoE-driven decision-making at various subsystems of a web backend (\eg per-machine resource allocation, scheduling, replica selection).
%Through developing novel algorithms and architectural components, we show that a {\bf QoE-driven web backend} (Figure~\ref{fig:intro-overview}(b)), which is aware of and embraces the differences of QoE sensitivity across requests, can substantially {\em improve the resource/QoE tradeoffs} of web backend; \ie better QoE without using more resources, or saving resources without degrading QoE. 
% Note that being QoE sensitivity does not require expensive infrastructure changes (\eg adding hardware or changing software stack).

\mypara{Research plan}
We divide the proposed research into three tasks.
%We use the following roadmap to thoroughly examine the benefits and challenges of QoE-sensitivity-aware web service backend.

\begin{packeditemize}
\item{\bf Quantifying potential benefits (Task \#1).}
We will do measurement studies to quantify the improvements brought by the QoE-driven web backend in the wild. It will include the opportunities of QoE-driven optimizations in web backend systems, and use dataset collected from large-scale web services to understand the potential in the real-world traffic patterns.

\item{\bf Estimating per-request QoE in real time (Task \#2).}
We will develop new system components, including tracing infrastructure and QoE prediction models, to estimate the impact of backend delay on requests' QoE in real-time. We explore the possibilities of incremental deployment by reusing the existing tracing and telemetry infrastructure in today's web backend systems.

\item{\bf QoE-driven control algorithms (Task \#3).}
We will develop novel QoE-driven control policies for web backend, including resource allocation, scheduling, and replica selection. Our design goals are (1) that the policies should achieve near-optimal QoE with minimal decision-making overhead, and (2) that their implementation should be amenable to existing systems.

%\item{\bf Impact on QoE fairness (Task \#4).}
%Finally, we will explore appropriate definitions of fairness to help strike a desirable balance between QoE-driven optimization and QoE fairness. This would also help us recognize potential threads of other systems/users taking advantage of the QoE-driven policies of the backend.

\end{packeditemize}


%\jc{why these applications?}

%\jc{Common challenges! getting QoE sensitivity, fairness definition!}

\mypara{PI qualification}
The PI's expertise includes computer networking, Internet QoE, and data analytics systems.
He has published 11 peer-reviewed research papers (6 first-authored) in top-tier networking and system conferences (\ie SIGCOMM, NSDI, CoNEXT).
More importantly, the PI has a deep understanding of Internet QoE. His doctoral dissertation, titled ``Enabling Data-Driven Optimization of Quality of Experience in Internet Applications'', is among the first systematic studies to apply data-driven approach to improving Internet  QoE. The dissertation won the CMU SCS Doctoral Dissertation Award and was nominated for ACM Dissertation Award.
During his PhD and postdoctoral years, he has extensive collaborations with Microsoft Research, Conviva Inc., and Google Research. These strong connections will help the proposed research to gain insights from the industry and provide viable paths to deployment.


% \vspace{0.2cm}
% \noindent{\em Thrust \#1: How much potential benefit can we get?}

% \vspace{0.2cm}
% \noindent{\em Thrust \#2: How to re-architect web services to be QoE-aware?}

% \vspace{0.2cm}
% \noindent{\em Thrust \#3: How to propagate user-perceived QoE information?}

%- This project proposes to re-architect the web service backend by making it QoE-aware. Our roadmap has three steps.\\
%1. XXX\\
%2. YYY\\
%3. ZZZ



% \vspace{2cm}
% User-perceived quality of experience (QoE) is one of the driving forces behind the Internet ecosystem, which consists of {\em subsystems}, \eg datacenters, CDNs, cellular carriers, backbone networks, content providers, who share resources across users. 
% % End-to-end Quality of Experience (QoE) is the driving force behind today's Internet application ecosystem, which includes several subsystem
% % The Internet application ecosystem consists of many subsystems, Cloud, ISP, CDNs, etc, and 
% Thus, one fundamental question is {\em how to share resources across users in a way that optimizes their overall QoE?}
% The primary constraint is that these subsystems are {\em federated}: it is impractical to orchestrate a global optimization where they relinquish the control on how their resources are shared. 
% Instead, the conventional wisdom has been that each subsystem shares its resources among users in a way that optimizes the overall performance metrics within its limit and imposes no differentiation between users if they are ``functionally'' identical (\ie same service, business relationship, etc).

% In contrast, we are driven by a simple observation derived directly from the federated nature of the Internet ecosystem.
% In a subsystem, there is {\em a substantial heterogeneity} among its users with respect to how sensitive their QoE is to the performance of the subsystem, even if these users are functionally identical. 
% Thus, the right question to ask is {\em not} how a subsystem should optimize the overall performance among users; instead, it should minimize {\em overall impact on user-perceived QoE}, which requires treating users differently, rather than equally, depending on how much impact it has on the user's perceived QoE.

% In this proposal, we apply this idea to improving QoE of web services.

% \mypara{Research goals}

% \noindent {\bf Intellectual Merit.~~}
% This proposal applies this idea in the context of cloud services. 
% \jc{what it means to cloud services? requests are going to be treated differently, etc} 
% Specifically, this idea can be applied to many services inside a cloud. \jc{talk about more applications}
% In this project, we plan to answer three key question:

% First, how much potential benefit does this idea have?

% Second, how to design a QoE-aware cloud scheduling/resource allocation mechanism?

% Third, how to propogate QoE information from users to the cloud?

% \noindent {\bf Broader Impacts.~~}


% \noindent {\bf Keywords.~~} 



% QoE matters to everyone!

% \subsection{Missed Opportunities}
% \begin{itemize}

% \item Today's tenant: every user should be treated with the same performance goal. Implicit assumption is that the impact of a subsystem is the same on all users.

% \item However, the federated architecture means:\\
% 1. QoE can be affected by any subsystem\\
% 2. Each subsystem serves users with different QoE sensitivities.

% \item Fundamental mismatch: some users who are less sensitive to the subsystem get over-optimized, while others who are more sensitive to the subsystem get under-optimized.

% \item New approach: minimize the overall impact on QoE. 

% \end{itemize}

% \subsection{This proposal: Making Cloud QoE-Aware}
% \begin{itemize}

% \item How the cloud works today -- agnostic to QoE

% \item QoE curve

% \item Examples of how things can be done differently!

% \end{itemize}


% \subsection{Research Roadmap:}
% \begin{itemize}

% \item How much potential benefit does this idea have?

% \item How to design a QoE-aware cloud scheduling/resource allocation mechanism?

% \item How to propagate QoE information from users to the cloud?

% \end{itemize}

\end{comment}

%!TEX root = main.tex

\section{Motivation and Framework}
In this section, we setup the context by introducing how today's web backends optimize QoE.
Then we show the observation that there is a great heterogeneity in the sensitivity of QoE to backend delay across requests, which inspires a new QoE-driven approach to web backend design.
%Then we give the empirical observation that there is great heterogeneity in the sensitivity of QoE to backend delay across requests, which inspires new opportunities to improve the QoE/resource tradeoffs of the web service backends without adding new resources. 
%Finally, we give a framework that contrasts today's approach with our proposed one.

\subsection{Today's approach}
% - Architecture of web service
%\mypara{A canonical architecture} 
\begin{wrapfigure}{r}{0.43\linewidth}
	\centering
	\includegraphics[width=0.95\textwidth]{figs/background.pdf}
	\caption{High-level architecture of today's web service and the lifecycle of a web request.}
	\label{fig:background}
\end{wrapfigure}
Figure~\ref{fig:background} depicts the high-level architecture of large-scale web services, and the lifecycle of a web request.
When a web request is issued by a user (\eg loading of a front page), it typically involves three components---client software, wide-area networks (WANs), and the backend---before the response is showed to the user.
The delay of each component contributes to the end-to-end delay and thus the user-perceived QoE (assuming the requested data is not cached locally by the client/ISP, etc). 
%In today's federated Internet architecture, the web service provider typically only has the control over the backend-induced delay. 
%Note that although client-side browsers/apps are developed by web service providers, the client-side delay is largely decided by how OS share resources among applications. 
A web service backend can be viewed as solving a resource sharing problem: optimizing the performance of many requests using a limited amount of resources. 
With the web service providers only controling the backend and the backend-induced delay\footnote{Although client-side browsers/apps are developed by web service providers, the client-side delay is largely decided by how OS share resources among applications.}, they seek to minimize the backend delay. 


% - Optimization metrics of each components: not QoE specific!
%\mypara{Today's approach}

While they measure the backend delay in different forms, today's backend systems share a common assumption (though often made implicitly) that the backend delay has the {\em same effect} on any request; \eg a delay of 20ms reading data from a database degrades the QoE by the same amount for any request . 
Under this assumption, it is sensible to measure performance by mean or percentiles (\eg 99$^\textrm{th}$ percentile) of backend delay or percentage of requests whose delays exceed some threshold (\eg service-level agreement).

Here, we only consider requests of the same application-level type, \eg content genre, user subscription type.
Note that even under the assumption that backend delay impact QoE in the same way for any request, requests can still be treated differently. For instance, in deadline-driven congestion control~\cite{??}, when congestion occurs, a small number of randomly picked flows might be delayed so that most flows can meet the deadline of flow completion time. 

\subsection{Key insight: Heterogeneity of QoE sensitivity}
%\jc{throw in some concrete improvement numbers}

Contrary to the assumption of backend delay having equal impact on QoE for any request, our key observation is that 
{\em the sensitivity of QoE to the backend delay varies across requests.}
%Our key observation is that the impact of backend delay on the QoE of a request actually depends on how much non-backend delay (Figure~\ref{fig:background}) the request experiences, including the client-induced and WAN-induced delays. 
%In other words, {\em the sensitivity of QoE to the backend delay varies across requests.} 
%\mypara{Why QoE sensitivity varies}

\begin{wrapfigure}{r}{0.5\linewidth}
	\centering
	\includegraphics[width=0.95\textwidth]{figs/qoe-curve.pdf}
	\caption{QoE curve showing the relationship between QoE and the end-to-end delay of a request.}
	\label{fig:qoe-curve}
\end{wrapfigure}
Figure~\ref{fig:qoe-curve} illustrates the intuition behind this observation using a dataset of \fillme web requests. The requests are from one of the Alex top-50 webpage, thus avoiding the impact of content on QoE. The requests were recorded in a period of 24 hours from users in \fillme countries, thus representing a decent coverage over space and temporal diversity.
The figure shows the relationship between QoE and the end-to-end delay of a request, which we refer to as the {\em QoE curve}.
%It is based on \fillme requests to a particular page (\fillme), thus avoiding the impact of content on QoE. 
For each request, we use the duration between the request is completed and the user leaves the web site domain (subsequent clicks from the page are counted as part of the session too) as the QoE metric. 
User engagement, like session duration, is typically used as subjective way of measuring QoE~\cite{engagement}.
The page load time is calculated by subtracting when the page is completely loaded by when the request is issued. 

The QoE curve is non-linear (sigmoid-shaped).\footnote{This curve may vary with the nature of the web sites and the requests (\eg QoE in general is less sensitive to delay when the user really needs to see the content of the page than when the website tries to entice users to do something), yet the sigmoid-like shape remains.}
This means that same amount of increase of page load time will have different effect on the QoE, depending on the position of starting point on the curve.
It can be intuitively explained. 
Users are not sensitive to a small change in delay when the overall delay is very short (below roughly 1200ms), because users typically gets annoyed by the delay only when it exceeds some threshold, or too long (above 2200ms), because the user has mostly tired of waiting. Only when the total delay is in the middle (roughly between 1200ms and 2200ms) can some change make a significant difference. 
%Secondly, we observe that the non-backend delays (client- and WAN-induced delays) of the requests are not concentrated in certain segment of the curve; rather, they spread out over the curve.
%In our dataset, we found \fillme\% requests have non-backend delay are below 1200ms, \fillme\% between 1200ms and 2200ms, and \fillme\% above 2200ms. 
%Combining this observation with the non-linear relationship between delay and QoE, we see that the sensitivity of QoE to the backend delay varies among requests. 


%\myparaq{What is new}
Indeed, we are not the first to observe the heterogeneity in non-backend delay across requests~\cite{timeciard,dqbarge} and the non-linear delay-QoE relationship~\cite{d3tcp, mun chiang's work}---in fact, the assumption that application flows have deadline utility function can be viewed as a simplification of the non-linear delay-QoE relationship.
What is new is their corollary that requests have different QoE sensitivity to backend delay, which, as we will see next, has a profound implication on QoE optimization. %how resource consumption and QoE optimization should be balanced.
%if the backend can add a total backend delay of $\delta=\delta_a+\delta_b$ to the two requests ($A$ gets $\delta_a$ and $B$ gets $\delta_b$), traditional methods will give both $A$ and $B$ a delay of $\delta/2$, whereas better QoE can be achieved by letting $\delta_a<\delta_b$. 
%We will give more concrete examles in \S\ref{sec:design}.

\subsection{New opportunities}
At a high level, the heterogeneous QoE sensitivity means that rathter than treating all requests as having equal sensitivity to the same amount of backend resources, the backend should instead allocate idle resources to requests that are more sensitive to backend delay.
To see it in action, suppose that request $A$ with non-backend delay $x_a$ is more sensitive to the backend delay (the red area of Figure~\ref{fig:qoe-curve}) than request $B$ with non-backend delay $x_b$ (the green or yellow area of Figure~\ref{fig:qoe-curve}). Now, if the backend resource allocation is such that one of them will get a backend delay of $\delta_1$ and the other one $\delta_2$, $\delta_1>\delta_2$. A QoE-agnostic method may make a random decision (\eg 50\% probability $A$ will get an end-to-end delay of $x_a+\delta_1$), but a QoE-driven method should instead assign the smaller backend delay $\delta_2$ to $A$, because $A$ is more sensitive to backend delay.

This simple example illustrates that the opportunities of QoE-driven depend on two factors.
First, both requests whose QoE is sensitive to the backend and those whose QoE is not must be present. As shown in Figure~\ref{fig:qoe-curve}, we found 41.5\% requests have QoE that is sensitive to the backend delay (red region between 1200ms and 2200ms), and the rest (29.8\% in the green region and 28.7\% in the yellow region) have QoE that is not as sensitive to the backend delay.
Second, requests should have different backend delay, which distributed systems naturally have for a number of reasons, \eg jobs in a queue will naturally have different waiting time, or replicas will naturally have different performance since they have different load.

%\jc{add inefficiency of non-QoE-aware approaches: 
%resources wasted, packing qoe-sensitive users with non-sensitive users}

\subsection{Framework}
\label{subsec:framework}
\begin{wrapfigure}{r}{0.45\linewidth}
	\centering
	\includegraphics[width=0.95\textwidth]{figs/framework.pdf}
	\caption{}
	\label{fig:framework}
\end{wrapfigure}
Next we present a high-level framework (Figure~\ref{fig:framework}) of transforming today's web service backend with QoE-aware optimization.
% shows a conceptual view of a QoE-aware web service backend. 
A typical web service backend has multiple subsystems, \eg messaging system, distributed database, inter-datacenter networks, and any delay caused by any subsystem will contribute to the backend delay of a request.
Although these subsystems are owned by the same organization, today there is little orchestration across these subsystems to jointly optimize for each request. 
It is beyond our scope to discuss the cost of separating these subsystems. 



This proposal takes a pragmatic stance. 
We optimize each individual subsystem by proposing minimal necessary changes in its existing ``control logic''.
A control logic can be implemented in different forms---\eg scheduling logic, resource allocation logic, congestion control sending rate, but it essentially share the limited resource among many requests.
\begin{packeditemize}
    \item The control logic of a QoE-aware subsystem takes in as part of its input the QoE sensitivity of any incoming request---an estimate on its non-backend and the corresponding position on the QoE-latency curve. 
    We envision different ways in which this information would be provided; \eg through a service that predicts the non-backend delay of a new request by profiling the non-backend delay of similar history requests, or adding a field in the request for any subsystem to tag the time it leaves each subsystem. (More discussion in \S\ref{sec:arch}.)
    \item Then, the control logic needs to be updated so that it can utilize the discrepancy of QoE sensitivity among requests to strike better QoE/resource tradeoffs (\ie the benefits we discussed in \S\ref{subsec:opportunities}).
    The contrast between the proposed QoE-aware control logic and traditional control logic can be explained as following: 
    For each request $r$ in a set of requests $R$, let $t_{r}^{(nb)}$ denote the non-backend delay, and $t_{r}^{(b)}$ the backend delay, and $Q(t)$ the QoE of the total backend and non-backend delay $t$.
    Traditional logic seeks to minimize average backend delay, \ie
    \begin{align}
        & \min \sum_{r\in R} t_{r}^{(backend)} \label{eq:old}
    \end{align}
    whereas, QoE-aware logic seeks to minimize average degradation of QoE, \ie
    \begin{align}
        & \min \sum_{r\in R} Q\left(t_{r}^{(nonbackend)}\right)-Q\left(t_{r}^{(nonbackend)}+t_{r}^{(backend)}\right) \label{eq:new}
    \end{align}
\end{packeditemize}




\section{Quantifying Potential Benefits in the Wild}
\begin{task}
We will carry out measurement studies to quantify various benefits---in QoE and resource---of a QoE-aware web service backend.
\end{task}

%The heterogeneity of QoE sensitivity across requests opens up many new opportunities to strike better resource/QoE tradoffs. 
In this section, we provide a taxonomy to categorize various possibilities of using the heterogeneous QoE sensitivity to improve the web service backend.%, and show simple-yet-realistic motivating examples to highlight two concrete use cases.

\subsection{A taxonomy of improvement}

%\mypara{Benefits}
We categorize the performance advantages of a QoE-driven backend over today's backend systems in four types:

\mypara{Improving QoE} 
First and foremost, a QoE-driven backend can achieve better QoE than a traditional one, by allocating resources in a way that more explicitly optimizes the total QoE. 
Here, we use a simple trace-driven simulation to quantify this potential improvement. 
Take Figure~\ref{fig:}
%In contrast to today's web backends which focus on the mean (or percentiles) of backend delay, we argue that they should instead {\em minimize the impact of backend delay on QoE}, so more resources (\eg higher priority) should be given to requests whose QoE is more sensitive to the backend delay. 
    
\mypara{Mitigating the impact of performance outliers} 
A common challenge of web service backends is how to trim the performance outliers---requests that experience excessively long backend delay, \eg due to server garbage collections. 
Rather than trying to eliminate outliers, the QoE sensitivity offers a new approach to {\em reduce the degradation on QoE due to outliers}.
Then instead of spreading requests evenly across servers, we can profile the performance of servers and predict the servers that are likely to have outliers~\cite{ganesh's trimming}, and assign requests with less QoE sensitivity to these servers, and keep requests with higher QoE sensitivity away from these servers. 
    
\mypara{Resource savings}
Similarly, one can reduce resource consumption, and possibly save energy, by spending less resources to optimize requests of lower QoE sensitivity. For instance, this can be done by deprioritizing the low QoE-sensitivity requests or sending them to servers with higher load, 
Doing so may be seen as unfair, but as we will discuss in \S\ref{sec:arch}, this can still be fair in the sense that each request gets an equal treatment (degradation) on the QoE.
    
\mypara{Cost-efficient explorations}
Finally, to cope with dynamic resource availability, backend systems (\eg CDN) maintain the visibility of many decisions (\eg edge clusters), by sending a fraction of request to randomly explore suboptimal decisions or injecting active probing traffic.
Neither approach is ideal. 
We take a different approach---one can use the requests of low QoE sensitivity to probe the suboptimal decisions, as long as the QoE degradation does not exceed that of sending a request of high QoE sensitivity to the optimal decision.
This way, we can maintain visibility with less impact on QoE.
%\end{packeditemize}
Note that these benefits can be realized without expensive adding more hardware or changing software stack, and can be done incrementally (they have no inter-dependency).

\mypara{Use cases}
These benefits can manifest themselves in many subsystems (Figure~\ref{??}) throughout a web service backend.
\begin{packeditemize}
\item {\em Scheduling:}

\item {\em Replica selection:} 

\item {\em Congestion control:}

\end{packeditemize}

\subsection{Proposed research}

- Quantifying the QoE-delay curve in the wild

- Quantifying the heterogeneity of QoE sensitivity

- Impact of QoE-aware logic on QoE (maybe delay is too small to be noticed anyway)





\section{Designing QoE-Aware Web Service Backend}
\label{sec:design}

\begin{task}
We will develop new scheduling and replica selection policies that leverage the heterogeneity of QoE sensitivity across requests to improve user-perceived QoE.
\end{task}

\subsection{Why new algorithms are needed?}

\mypara{Message scheduling}
% - fifo and etc don't work
To show the need of a new policy for QoE-driven  scheduling, we consider a simple example (shown in Figure~\ref{??} ) in which there are three identical requests (\ie they are of the same priority) in the queue but they have different non-queuing delays.
We compare three policies: 
(1) FIFO, which is the default policy for equal-priority requests.
(2) Earliest-Deadline-First (EDF), which uses a fixed deadline on the sum of queuing delay and non-queuing delay and prioritizes requests whose non-queuing delay is closest to the deadline, and 
(3) the optimal scheduling policy (OPT) that maximizes average QoE.
We assume that the page load time of a request is the sum of the non-queuing delay and the queuing delay (that is, the number of jobs ahead of it times the per-request processing delay).
Figure~\ref{??}  shows the outcome of running each policy. 
We can see that OPT outperforms FIFO, because it is agnostic to non-queuing delay
OPT also outperforms EDF with various deadlines, even though it somehow takes the non-queuing delay into account (by imposing a deadline on the total of queuing and non-queuing delay).
This is because how close a request's non-queuing delay is to the deadline is {\em still} agnostic to the sensitivity of QoE to the queuing delay. 
Remember in Figure~\ref{??}, the QoE sensitivity does not grow monotonically to the non-backenc(queuing) delay. 

\mypara{Web server replica selection}
% - example of tails
Large-scale distributed systems are susceptible to {\em performance outliers} on some replicas---\eg exceptionally long processing time on some requests due to CPU contention, software garbege collection, etc. 
Despite much effort devoted to eliminate performance outliers, they are still prevalent, resulting in bad tail (\eg 99$^\textrm{th}$ percentile) performance even if the mean/median performance is good.
Our first idea is that rather than cutting the performance outliers, we show that QoE-aware replica selection algorithm can {\em contain} the negative impact of performance outliers on user-perceived QoE.

Consider a system with two replicas, as in Figure~\ref{??}. 
Both replicas have the same load-performance profile that as load (number of requests received per second) goes up, the 99$^\textrm{th}$ percentile delay increases sharply, though the mean performance has only mild increase. 
This illustrative setting is simplified but in consistent with the load-performance profile in real datacenters~\cite{??}.
Suppose that the workload is \fillme requests per second, that the requests have same QoE-delay curve, and their non-backend delays follow the distribution show in the figure.
We compare two traditional replica selection policies---load balancing policy that strives to put even load, latency-aware policy~\cite{c3,cassandra,etc} that optimizes for Eq.~\ref{eq:old}---with an optimal policy that optimizes for Eq.~\ref{eq:new}.
We see that the optimal solution achieves better tail QoE than both baselines. 
Both load balancing policy and latency-aware put even load to the replicas, which make both equally overloaded, causing long tail delay and bad tail QoE.
In contrast, the optimal policy diliberately puts more requests on one than another, but does so by letting QoE-sensitive requests use the lighter-loaded replica and letting QoE-insensitive requests use the slightly more loaded replica. 

\subsection{Why challenging?}
The new QoE-aware scheduling policy should be {\em near-optimal} (\ie achieve QoE close to the optimal scheduling outcomes) and {\em efficient} (\ie constant decision time on each incoming request).
To see why it is challenging to meet both requirements simultaneously, let us consider two baselines.
On one hand, standard approximating\footnote{Unfortunately, optimzing a non-contex curve objective function, as in our case, requires running complex algorithms, whose runtime increases super-linearly with the number of variables~\cite{??}.} algorithms (\eg~\cite{??}) often requires non-constant overhead to compute the scheduling order of each request. 
On the other hand, fast scheduling algorithms can be suboptimal. 
For instance, one efficient scheduling algorithm is to give each request a priority that is proportional to the {\em slope} of its non-queuing delay on the QoE-delay curve. 
This makes sense, because if the request were to stay in the queue for a (short) fixed amount of queuing delay, this slop indicates exactly how much QoE degradation will be caused by the queuing delay, and the decision runtime is constant for each request.
However, the queuing delay does vary among requests as well as over time (as it might be pre-emptived by future requests), and the resulting QoE might suffer.
If the queuing delay is non-trivial, the slope of a request's non-queuing delay can be both an overestimate or underestimate of the acutal QoE sensitivity of a request. 
In Figure~\ref{??}, for instance, if the queuing delay is 200ms, a request with 1100ms non-queuing delay will be believed as insensitive to the queuing delay, while after 200ms queuing, it actually will see significant drop in QoE. 

Moreover, the new scheduling policy has to be {\em amenable} to the implementation of existing message brokers. 
For instance, in RabbitMQ, once a request is pushed to a FIFO queue associated with some priority, and cannot be easily changed to another FIFO queue. 
That means, the related ordering of two requests cannot be changed once they are scheduled.
For instance, the aforementioned slope-based scheduling algorithm cannot be easily fixed by re-assigning a priority that reflects the slope after the amount of queuing delay.
(Note this does not preclude the possibility of preemption by a future request if we assign it with a high priority.)

\subsection{Proposed research}
We propose to develop new scheduling policies that leverage the heterogeneity of QoE sensitivity across requests to improve user-perceived QoE. 
One potential idea is to rely on the fact that the number of requests and the distribution of their non-queuing delay is relatively stable at a timescale of several seconds. 
Thus, we can periodically recalculate the priority of a request with certain non-queuing delay by assuming that future requests will be drawn from certain distribution.
Then, the scheduling problem can save this mapping in a table, and determine the priority of any new request by looking-up the table. 




\section{A New Architecture, a New Way of Thinking}
\label{sec:arch}

This section focuses on two basic architectural questions behind QoE-driven backend---how to make web backend know of the QoE sensitivity and how should fairness be defined in our context.

\subsection{How to make web backend aware of QoE sensitivity?}
\begin{task}
We will explore new architectural component of web backend, including tracing infrastructure and QoE prediction models, to estimate the QoE sensitivity of requests in real-time.
\end{task}

\mypara{Estimating non-backend delay}
Making web backend aware of a request's QoE sensitivity is difficult in today's federated Internet architecture with the web backend system only having the visibility of the delay within its own system (Figure~\ref{??}), while QoE, defined by either subjective user satisfaction or end-to-end response time, is only observable by the clients themselves. 
We envision that the QoE-delay curve, \ie relationship between end-to-end delay and QoE, should be profiled offline (for each content/application), because it is not expected to change frequently.
If the backend knows the non-backend delay of any incoming request, it would then be able to derive the QoE sensitivity of the request from the QoE-delay curve. 
We envision two basic approaches to measure the non-backend delay of a new request. 
\begin{packeditemize}
\item {\em History-based:}
One approach is to use the non-backend delay of history requests to build a prediction model that given any new request, can predict the non-backend delay. Once a request is completed, the web site or content provider will know its end-to-end delay (\eg via client-side instrumentation), and by matching the end-to-end delay with with get the non-backend delay with the backend delay measured by the backend, we can get the non-backend delay of this history request.

\item {\em Tracing-based:}
Another approach is directly estimate the non-backend delay of a request using the tracing infrastructure on the client and backend to directly measure the delay between when the client issues the request and when the backend receives it, \ie the ``incoming half'' of the non-backend delay. Prior work has found that this incoming half of the non-backend delay is indicative of (though not equal to) the ``outgoing'' half of the non-backend delay~\cite{timecard}.
\end{packeditemize}
Note that the two approaches represent a tradeoff---the history-based approach requires no additional real-time tracing of a new request and thus less intrusive to the existing system, whereas the tracing-based approach is more intrusive but might be more accurate as it directly measures the request under consideration.

\mypara{What pieces in today's systems can be used}
In fact, the tracing infrastructure, similar to what is needed for the tracing-based approach, is already available in several large-scale web service infrastructures. Facebook has introduced such a tracing system~\cite{mysterymachine}. Microsoft has a similar internal tracing system that keeps track of several key timestamps of a request (\eg when is issued by client, received by the edge web server, received and processed by a backend database)~\cite{sid's paper}.

\mypara{Proposed research}
%- Cross-subsystem tracing and predicting non-backend delay
While both approaches sound promising, we found that they may not meet the requirement of estimating QoE sensitivity of requests in real-time. 
These tracing systems and QoE recording systems are mainly built for {\em offline} diagnosis, rather than online inference. As a result, these timestamps are often recorded by individual subsytems in a distributed fashion, but only collected to a central place that can be queried every several minutes or even hours. 
To address this problem requires a better solution. 
One possibility is to make these tracing information (\eg timestamps of the past events of a request) ``in-band'' as part of the request itself. This may require changes in the message format or client-facing interfaces, but will make these tracing information instantly available when the backend recives the request.

\subsection{How should fairness be defined?}
\begin{task}
We will investigate the ``price of fairness'', \ie loss in QoE in exchange for more fairness, and how to define fairness that is robust to actions of other systems.
\end{task}

One evident downside of QoE-driven control policies is the seemingly unfair treatment of priorizing requests with higher QoE sensitivity over others, even though these requests are of the same nature (same application, same content, etc).

\mypara{This might be fair after all}
At first glance, giving more resources to one request than another, just because one request spends less (or more) time over the WAN, does not seem fair.
However, if we take the ``utility'' of these resource into account, such behavior could be viewed as perfectly fair under some circumtances.
More specifically, from the perspective a client, what matters is not how much resource her request is being process with; what matters is how much {\em QoE degradation} a particular resource allocation causes.
To use the  terminology  introduced in \S\ref{subsec:framework}, we can define fairness based on standard fairness index (\eg Jain's index~\cite{??}) over the QoE degradation, \ie $Q\left(t_{r}^{(nonbackend)}\right)-Q\left(t_{r}^{(nonbackend)}+t_{r}^{(backend)}\right)$, among requests $r\in R$.
More importantly, achieving this notion fairness can be directly incorporate in the solving the Eq.~\ref{eq:new} as a constraint.
%To use the terminology used in \S\ref{subsec:framework}, we can define a max-min fairness as
%\begin{align}
%& \min\max_{r\in R}Q(t_{r}^{(nb)})-Q(t_{r}^{(nb)}+t_{r}^{(b)}) \label{eq:fairness}
%\end{align}
%Under the Nash standard, a transfer of resources between two players is favorable and fair if the percentage increase in the utility of one player is larger than the percentage decrease in utility of the other player. 
%In our context, the Nash standard means that our resource allocation is favorable and fair, as long as the any transfer of resource will result in larger percentage of decrease in the QoE of one request than the percentage of increase in the QoE of another request.

%\mypara{Fairness under cross-entity settings}
%However, achieving fairness can be more difficult, if multiple subsystems (\eg message broker and web server selection) in web backend make QoE-driven decisions {\em independently} .
%Consider a backend of two consecutive subsystems $X$ and $Y$, \ie any request will go through $X$ and then $Y$ so that the backend delay is the sum of the two subsystems. 
%Now, suppose request $A$ has a greater non-backend delay than request $B$, which makes $A$ more sensitive to the backend delay than $B$. 
%So naturally, subsystem $X$ might process $A$ with greater backend delay than $B$. However, after considering this, 

\mypara{This can be taken advantage of}
However, achieving fairness can be tricky, if another system (\eg ISP) knows that the web backend allocates resources among request in the proposed way. In particular, the other system can trick the web backend into treating requests unfairly.
Consider a simple example where the end-to-end delay of a request is the delay of ISP $X$ and the delay of a web backend $Y$, and there  two requests $A$ and $B$.
Now, suppose ISP $X$ adds a small amount of delay to $B$ so that $B$'s QoE less sensitive to the backend delay than $A$. Following the principle of fair QoE degradation, backend $Y$ should treat $A$ and $B$ differently, giving more resource to $A$ than $B$. 
Note that $X$ would be able to do the same, if $Y$ defines fairness in a traditional sense (\ie with respect to backend delay, rather than QoE degradation).
While it seems oversimplified, this illustrates an underlying issue of QoE-driven web backend that by making the fairness metric and objective function linked with non-backend delay, we make the backend vulnerable to actions of other systems.

\mypara{Proposed research}
We will investigate the classic tradeoff between fairness and utility in the context of QoE-driven web backend. In particular, is it possible to solve Eq.~\ref{eq:new} in a near optimal fashion, as discussed in \S\ref{sec:design}, while maintaining, say max-min fairness, or there will be a price in fairness for better QoE.
In the meantime, we notice that by being QoE-agnostic, traditional web backends have already induced some unfairness in terms of user-perceived QoE. It would be helpful to compare our price of fairness with that of a traditional web backend, and see if we can still get sufficient QoE improvement by paying the same amount of price of fairness.

Finally, we will investigate how fairness should be defined so that we can avoid being negatively affected by other systems diliberately misleading our backend into treating requests unfairly. One potential solution is to monitor the delay of non-backend systems (\eg ISPs) of history requests in order to detect whether they add artificially add more delays; when they do, the backend then should stop inferring the requests' sensitivity by their actual non-backend delay; and instead assume the non-backend delay is at the normal level.


\section{Related Efforts}
Some earlier work also takes non-backend delay into account when optimizing cloud-based applications. 
Timecard~\cite{timecard} allows requests that have shorter network delay to wait longer for the web server to retrieve better ad content. 
DQBarge~\cite{dqbarge} similarly trades delay for better response quality by monitoring the progress of parallel subtasks and allowing faster subtasks to process longer to get better results without increasing the total response time.
Also relevant are works that dynamically predict the response-time service-level agreement of cloud applications~\cite{Rich Wolski}, which are driven by the same need to cope with the heterogeneity of non-backend delay across users and over time.
Our previous work EONA~\cite{eona} pioneers the idea of driving individual systems of the application ecosystem (including web service backend) by user-perceived QoE.
\jc{Difference to application-system joint optimization (\eg coflow)?}
However, none of these works show the benefits of QoE-awareness through the lens of resource/QoE tradeoffs. 

\jc{diff to QoE management. see wiki page.}





% \subsection{Categorizing the benefits}
% \begin{itemize}

% \item Mean performance

% \item Cutting tails

% \item Resource savings

% \item Cheap exploration

% \end{itemize}

% \subsection{An Example Using Trace-Driven Simulation}
% \begin{itemize}

% \item The QoE Curve

% \item Application (Messaging layer)

% \item Synthetic trace generation

% \item Results

% \end{itemize}

% \subsection{Research plan}

% \newpage
% \section{Architecting for QoE-aware cloud}

% \subsection{Theoretical formulation}

% \begin{itemize}
% \item 
% \end{itemize}

% \newpage
% \section{Propogating QoE information}


%!TEX root = main.tex
\section{Intellectual Merits and Broader Impacts}

%- Intellectual merit: A fundamental view of QoE optimization: taking an end-to-end view. potentially changing the way other internet services should be operated.
\mypara{Intellectual merits}
The proposed research will address interdisciplinary challenges and take an integrative approach to design and implement novel QoE-driven web services along four key thrusts: 
1. Quantifying the potentials of QoE-driven web backend to improve web QoE and backend resource utilization in the wild;
2. Developing QoE-driven control logic for common tasks in web backend systems, including message scheduling and replica selection;
3. Developing new architectural components to make web backend aware of the QoE sensitivity;
4. Exploring the possibilities of QoE-driven optimization negatively affecting fairness.
The key insight is a fundamentally new approach to QoE optimization over the federated Internet architecture, which argues that individual component should not be treated in isolation; instead, they should be made aware of the opportunities naturally inherent in the fact that end-users are affected by multiple components. 
This bears implications not only for web backend, but many other systems too, such as content delivery networks, cloud providers, ISPs, cellular carriers, etc.

%- Broader impact: good for industry. achieving better quality/cost tradeoffs. teaching web services, quality of experience
\mypara{Broader impacts}
The research carried out as part of this proposal will have key implications for academia, industry, and education.


\begin{packeditemize}
\item {\em Research impact:}
The proposed research will result in broad academic.
The PI will publish the results from the proposed research at top tier conferences and journals spanning both systems (e.g., SOSP, OSDI) and networking (e.g., SIGCOMM, NSDI, CoNext, IMC).
The PI has strong track record of publishing at these venues.
% dissimination

\item {\em Education:}
The PI will develop new course materials and integrate it into his course offerings in systems and networking. 
% cloud

\item {\em Impact on industry:}
The proposed work has significant implciations for several commercial service providers, including content delivery networks, application developers, and cloud-hosted web service providers.
This work will inform the critical industry evolution by  web service providers to more QoE-driven approaches.
% add conviva, microsoft, google

\item {\em Societal impact:}
By delivering dramatic improvement in the experience of users of Internet applications and the efficiency of web service infrastructure, the research will have broader social impact. 
% improve qoe increase the engagement of users. resoruce savings  

\end{packeditemize}


%%!TEX root = main.tex

\section{Introduction}

\noindent A key challenge facing Internet services today is how to share resources across users to maximize the user-perceived {\em QoE} (Quality of Experience) by minimizing their end-to-end delay.
Maintaining desirable level of user-perceived QoE is critical, with reducing a few hundred milliseconds from the page load time means millions of dollars.
%The fundamental challenge facing large-scale web service providers and (\eg Microsoft, Facebook, Akamai, Google, Amazon) is how the backend systems should share its resources in order to optimize user-perceived {\em QoE} (Quality of Experience). 
%Maintaining desirable level of user-perceived QoE critical for their revenue models.
A delay penalty of 400ms in Google search responses reduces search volume by 0.74\%; and 500ms of latency for Bing decreases revenue by 1.2\%~\cite{google-revenue,bing-revenue}; For Amazon, an additional latency of 100ms means a 1\% drop in sales~\cite{amazon-revenue}.
Yet, despite substantial efforts (\eg~\cite{shandian,gaze,rosen2017push,jalaparti2013speeding}), maintaining desirable QoE remains a challenge, with average page loading time of Facebook requests being over 3000ms~\cite{mystery}.
% with 50\% users of some popular website spending over 30\% of page loading time on the web backend~\cite{mystery}.

We argue that the key missing piece of today's Internet architecture is that individual Internet systems, \eg web backends, CDNs, and ISPs, are agnostic to their impact on individual users' real-time QoE, especially the difference in their impact across users.
As a consequence, it is difficult for these systems to direct optimizes QoE. 
Instead, our overarching thesis that these systems should be driven directly by the real-time information about their impact on the QoE across users, which could substantially improve QoE as well as resource efficiency, without adding new resources.
While there have been similar efforts in the past (\eg~\cite{alto,frank2013pushing,xie2008p4p,jiang2009cooperative}) toward similar end-to-end quality optimization, our project is inspired by several favorable recent trends, including ``use-case pulls'', \eg the prevalence of QoE-driven revenue models (\eg~\cite{akamai-report,dobrian2011understanding}), as well as ``technological pushes'', \eg the wide use of tracing from the clients and the backend systems (\eg~\cite{mystery,zhao2014lprof}). 

This proposal explores the benefits and challenges of our thesis in the context of how web backend systems should optimize web QoE by leveraging user heterogeneity. 

\mypara{Limitation of today's web backend}
%The web backend systems today have no direct visibility of the QoE of each web request when it arrives, 
Because the end-to-end delay (page loading time) of a web request is affected by the delay of many non-backend systems (ISPs, client-side software, etc) which is beyond the scope of the web service providers, the web backend systems have no direct visibility of how much impact it has on the QoE of individual requests.
As a result, the web backend systems seek to minimize the overall {\em backend delay} (\eg the mean, tail values, or the probability of missing some SLA deadline), under the assumption that backend delay of $n$~ms has the {\em same} impact on any request.\footnote{Modulo the content-specific (\eg web page type) or user-specific (\eg free vs. premium subscription) factors.}
%Page load time, which we refer to as {\em end-to-end delay}, generally consists of three parts: client-side delay, wide-area network (WANs) delay, and backend delay.
%- Web services, like applications running in the cloud, have been basing their optimizations on the goal of improving server-side latency (sometimes the fraction of users meeting some fixed deadline)
%Because of the federated nature of Internet architecture, web service providers do not have full control over all types of delays.
%---to them, WANs and clients devices are largely blackboxes operated, not by the web services, but by ISPs, cellular carriers, and device vendors.
%(while web browsers and apps are developed by the web service providers, the client-side performance is largely decided by how OS share resources among multiple applications).
%With web service providers only controlling the web backend, the performance metric they focus on optimizing is the 
%%Thus, instead of optimizing for QoE directly, today's web services focus on reducing the 
%%different requests have the same {\em QoE sensitivity to backend delay}
%{\em backend delay}, under the assumption that a backend delay of $n$~ms has the {\em same} impact on any request.\footnote{Modulo the content-specific (\eg web page type) or user-specific (\eg free vs. premium subscription) factors.}
% That is, a backend delay of $n$~ms has the same effect on the QoE of any request.
%For instance, they minimize the mean/tail backend delay or the fraction of requests whose backend delay exceeds some deadline (\eg 300ms)~\cite{??,??}.
%- This project takes a step back and asks a different question: does the latency have the same impact on user QoE? 
% In doing so, 
%all requests are optimized with the same objective function of backend response time; 
% an implicit assumption is that different requests have the same {\em QoE sensitivity to backend delay} (modulo content-/user-specific factors, such as web page type or subscription type, etc);
% that is, a backend delay of $n$~ms has the same effect on the QoE of different requests.

We take a step back, and ask {\em ``does the backend delay really have the same impact the QoE of any web request?''}
%- The answer is no, which has profound impact on how web services should be built. [Give a simple example here.] In essence, this means giving each ``priority'', in terms of resources and scheduling, is cost-inefficient and suboptimal. [Give a simple example. resources wasted for users who are screwed already]
Our answer is {\em no}, and such heterogeneous sensitivity of QoE to backend delay is pervasive among requests of the same application. 
This is due to the fact that how much impact the backend delay has on a request depends on the non-backend delay experienced by a request (\eg ISP routing, client-side resource allocation by the device OS), which can have a great variability among requests and over time~\cite{timecard,dqbarge}.
%Two observations contribute to this conclusion: the non-linear relationship between page load time and QoE~\cite{??} and the fact that the WAN/client delay varies among requests~\cite{timecard,dqbarge}, {\em the QoE sensitivity to backend delay varies among requests.}
For instance, the QoE of a web request that has spent 50ms on wide-area networks tends to be more sensitive to 10ms backend delay than a request that has already spent 500ms on the network.
By falsely assuming requests are equally sensitive to the backend delay, traditional web backend system (Figure~\ref{fig:intro-overview}(a)) might waste resources on requests that are insensitive to the backend delay, and/or have suboptimal QoE (\eg using inadequate resources on requests whose QoE is critically dependent on the backend delay. 

\mypara{Our approach} 
This proposal introduces ``QoE sensitivity'' as a new dimension for web QoE optimization, and propose to develop {\bf QoE-driven web backend} systems (Figure~\ref{fig:intro-overview}(b)). 
We argue that embracing the heterogeneity in QoE's sensitivity to backend delay has profound implications for backend systems: 
it should allocate resources in a way that favors the requests whose QoE is more sensitive to backend delay to improve the overall QoE while having minimal drop in the QoE of other requests.
For instance, in a trace-driven simulation (\S\ref{sec:quantifying}), we found that a simple QoE-driven resource allocation policy can raise user experience (measured in user engagement) 50\% closer to optimal (\ie zero backend delay) than a QoE-agnostic baseline web backend.
%should allocate their limited resources. 
%By taking into account the QoE's sensitivity to the backend delay, 
%\jc{bring up some concrete improvement numbers}
%\jc{need to highlight that this is not because application differents}

\begin{figure}[t]
	\centering
	\vspace{-0.5cm}
%	\hspace{0.6cm}
	\includegraphics[width=0.75\textwidth]{figs/intro-overview-new.pdf}
	\vspace{-0.1cm}
	\tightcaption{We propose to re-architect (a) today's web backend which seeks to minimize the backend delays into (b) a QoE-driven web backend which seeks to minimize the impact of backend delay on QoE.}
	\label{fig:intro-overview}
\end{figure}

%\jc{give a figure to contrast optimization of backend in-isolation vs. QoE-aware.}

%- Research goal: This project proposes that the web service backend should be aware of the QoE sensitivity. This effectively changes how one formulates the web service optimization problem.


%The key difference is that 
%Unlike a traditional backend which seeks to minimize the backend delays, a QoE-driven backend seeks to minimize the impact of backend delays on user-perceived QoE.

Despite its promise, we face two key challenges.
First, when a request is received, the backend needs to estimate in real time the sensitivity of the request's QoE to the backend delay with sufficient accuracy, and propagate the information across individual components of the backend. This information is not readily available in today's backend.
Second, the resource allocation and scheduling logic of web backend subsystems need to be updated to leverage the heterogeneity across users. The challenge arises from the fact that requests need to be set with different priorities in real time depending on their non-backend delay.
There are other problem inherent to the QoE-driven approach, including control stability (\ie the backend system becomes sensitive to any changes in non-backend delay), and exacerbating QoE unfairness (\ie the backend system favors sensitive users who may already have better QoE than some others).
%Through developing novel algorithms and architectural components, we show that a {\bf QoE-driven web backend} (Figure~\ref{fig:intro-overview}(b)), which is aware of and embraces the differences of QoE sensitivity across requests, can substantially {\em improve the resource/QoE tradeoffs} of web backend; \ie better QoE without using more resources, or saving resources without degrading QoE. 
% Note that being QoE sensitivity does not require expensive infrastructure changes (\eg adding hardware or changing software stack).

\mypara{Research plan}
This proposal explores the problem space by focusing on addressing the first two challenges, and will also discuss other questions (\eg control instability and unfairness).
We divide the proposed research into three main tasks. 
%We use the following roadmap to thoroughly examine the benefits and challenges of QoE-sensitivity-aware web service backend.

\begin{packeditemize}
\item{\bf Quantifying potential benefits (Task \#1).}
We will use a mix of user studies and analysis of industry datasets to quantify the potential improvement (in terms of QoE and resource savings) brought by the QoE-driven web backend in real-world workloads and identify the opportunities of QoE-driven optimizations in existing web backend systems.%, and use measurement dataset from large-scale web services to understand its potential in the real-world traffic patterns.

\item{\bf Estimating per-request QoE in real time (Task \#2).}
We will develop new tracing infrastructure and QoE prediction models to estimate the impact of backend delay on requests' QoE in real time. We will investigate the possibilities of incremental deployment by reusing the existing tracing and telemetry infrastructure in today's web backend systems.

\item{\bf QoE-driven control algorithms (Task \#3).}
We will develop novel QoE-driven control policies for web backend, including resource allocation, scheduling, and replica selection. Our design goals are (1) that the policies should achieve near-optimal QoE with minimal decision-making overhead, and (2) that their implementation should be amenable to existing systems.

\end{packeditemize}


\mypara{PI qualification}
The PI's expertise includes computer networking, Internet QoE, and data analytics systems.
He has published 11 peer-reviewed research papers (6 first-authored) in top-tier networking and systems conferences (\ie SIGCOMM, NSDI, CoNEXT).
More importantly, the PI has a deep understanding of Internet QoE. His doctoral dissertation, titled ``Enabling Data-Driven Optimization of Quality of Experience in Internet Applications'', is among the first systematic applications of data-driven approach to improving Internet QoE, and some of the proposed QoE prediction techniques (\eg~\cite{cfa,c3}) have led to real-world deployment and impact. The dissertation won the CMU SCS Doctoral Dissertation Award and was nominated for ACM Dissertation Award.
%During his PhD and postdoctoral years, 
He has extensive collaborations with Microsoft, Conviva, and Google, which will help the proposed research gain insights from the industry and provide viable paths to deployment.
%These strong connection

This project will provide the material needed for further research on the QoE-driven systems and is part of the necessary and critical steps for the PI to achieve research independence.



%\newpage










\begin{comment}


\noindent The ecosystem of web applications critically depends on maintaining desirable user-perceived QoE (quality of experience).
% QoE depends critically on web page loading time, 
Yet, despite tremendous efforts,
%(\eg cutting tail latency via redundancy or pushing caches closer to end users), 
many users still suffer from suboptimal QoE.
%Unlike previous approaches such as cutting tails of response time or pushing caches closer to end users, 
One fundamental issue is that, due to the federated Internet architecture, it is difficult for the web backend to measure the impact of its delay on a web request's QoE in real time. As a result, today's web backends seek to minimize the delay of every request to the same level.

This proposal introduces a new dimension for optimizing web QoE, which has been ignored by today's web backend: embracing the {\em inherent heterogeneity} of how much impact the {\em backend delay} has on different users' QoE. Such heterogeneity results from not only different applications or services, but more importantly difference in the delay of non-backend systems (\eg ISPs).
%, even if they request the same type of application/service.
% {\bf QoE sensitivity to backend delay} across users. More importantly, such heterogeneity is prevalent even if the users request the same type of application/service.
%, \ie how sensitive a user's QoE is to the web backend delay.
For instance, a web request that has spent 50ms on wide-area networks tends to be more sensitive to 10ms delay of web backend than a request that has already spent 500ms on the network. 
Such discrepancy in QoE's sensitivity to backend delay has profound implications---web requests previously seen as indistinguishable by the backend can now be treated differently so as to improve overall QoE by favoring the requests whose QoE is more sensitive to backend delay without hurting other requests' QoE or adding any new resources.
% on how web backend should allocate its limited resources across requests. In essence, 
%We show early promising result that by making existing the web backend aware of QoE sensitivity, we could improve both QoE and resource efficiency than existing solutions.
To fully unleash the potential, we propose to re-architect today's web backend systems by elevating {\em the sensitivity of QoE to the backend delay} as a key factor in the control logic of web backend.
% investigate new opportunities to improve QoE and save resources by making web services aware of QoE sensitivity (\eg better scheduling policies and replica selection policies) and 
We face two key technical challenges: obtaining real-time QoE feedback as input to the backend systems, and QoE-driven algorithms that achieve better QoE/resource tradeoffs.
%balancing QoE and efficiency, estimating the QoE sensitivity, and addressing fairness issues.
We plan to quantify the potential of QoE-driven web backend in the wild, and present new system component to provide real-time QoE estimation, and novel QoE-driven scheduling and resource allocation policies for better QoE and resource efficiency.
% (1) We quantify its potential benefits in QoE and resource savings.
% (2) We propose novel algorithms for QoE-aware scheduling and resource allocation of web services. 
% (3) We present novel system designs and implement prototypes that make web services QoE-aware in practice.

A future generalization of the project is that today's ecosystems of applications (\eg web, video streaming) are such that it is difficult for individual systems (\eg CDNs, ISPs, cloud) to explicitly drive decision-making towards better QoE. 
Our overarching vision is that by driving these systems explicitly with user-perceived QoE, we can unleash new opportunities to significantly improve the resource-efficiency and QoE.
%not built to directly optimize user-perceived QoE, and we believe the key missing piece is to let every system (including CDNs, ISPs, web service backend) be aware of the impact of their control decisions on QoE. 
By enabling web backend systems to embrace heterogeneous QoE sensitivity, this proposal is a first step towards the vision. 


% Thus, the goal of each subsystems in a large web service, such as web server or key-value store, should be to optimize the overall QoE of many users given limited resources. 
% A common approach to achieving this goal is for each subsystem to optimize some ``local'' performance metrics measured within its scope (\eg server-side delay) over all users, and the intuition is that if each subsystem follows the approach, it will optimize the overall QoE of users. 
% We argue, however, that this approach only achieves suboptimal QoE and can use more resources than necessary. 
% Our key observation is that {\em the impact of a subsystem's performance on a user's perceived QoE varies greatly among users} (modulo web page type, business relationship), so when sharing resources across users, each subsystem should take into account its impact on each user's QoE.
% One typical sources 
% This has profound implication on how web services should be optimized, and opens up many several new opportunities.


\section{Introduction}

% - QoE is important and our goal is to improve QoE for Web Services.
\noindent The fundamental challenge facing large-scale web service providers (\eg Microsoft, Facebook, Google) is how the backend system should  share its resources across users to optimize user-perceived QoE (Quality of Experience). 
%Web QoE has been shown to be critically dependent on web page loading time~\cite{??,??}.
Despite substantial efforts~\cite{??,??,??} and more resources~\cite{??,??,??}, maintaining desirable QoE remains a challenge with 50\% users of some popular website spending over 30\% of page loading time on the web backend~\cite{mystery}.
%whose experience could have been improved from bad to good if the backend delay is zero~\cite{dqbarge}.
%Their business models, based on advertisement or subscription, are driven by user engagement, for which QoE is believed to play a vital role (among other factors such as content, user interfaces).

\mypara{Limitation of today's web backend}
We argue that a key reason is that the web backend systems do not have the direct visibility of the QoE of each web request when it arrives. 
This is because QoE is affected by the performance of many non-backend systems (ISPs, client-side software, etc), while the web service providers only control the web backend systems.
As a result, web service providers focus on minimizing the {\em backend delay}, with the assumption that backend delay of $n$~ms has the {\em same} impact on any request.\footnote{Modulo the content-specific (\eg web page type) or user-specific (\eg free vs. premium subscription) factors.}
%Page load time, which we refer to as {\em end-to-end delay}, generally consists of three parts: client-side delay, wide-area network (WANs) delay, and backend delay.
%- Web services, like applications running in the cloud, have been basing their optimizations on the goal of improving server-side latency (sometimes the fraction of users meeting some fixed deadline)
%Because of the federated nature of Internet architecture, web service providers do not have full control over all types of delays.
%---to them, WANs and clients devices are largely blackboxes operated, not by the web services, but by ISPs, cellular carriers, and device vendors.
%(while web browsers and apps are developed by the web service providers, the client-side performance is largely decided by how OS share resources among multiple applications).
%With web service providers only controlling the web backend, the performance metric they focus on optimizing is the 
%%Thus, instead of optimizing for QoE directly, today's web services focus on reducing the 
%%different requests have the same {\em QoE sensitivity to backend delay}
%{\em backend delay}, under the assumption that a backend delay of $n$~ms has the {\em same} impact on any request.\footnote{Modulo the content-specific (\eg web page type) or user-specific (\eg free vs. premium subscription) factors.}
% That is, a backend delay of $n$~ms has the same effect on the QoE of any request.
For instance, they minimize the mean/tail backend delay or the fraction of requests whose backend delay exceeds some deadline (\eg 300ms)~\cite{??,??}.
%- This project takes a step back and asks a different question: does the latency have the same impact on user QoE? 
% In doing so, 
%all requests are optimized with the same objective function of backend response time; 
% an implicit assumption is that different requests have the same {\em QoE sensitivity to backend delay} (modulo content-/user-specific factors, such as web page type or subscription type, etc);
% that is, a backend delay of $n$~ms has the same effect on the QoE of different requests.

We take a step back, and ask {\em ``does the backend delay really have the same impact the QoE of any web request?''}

%- The answer is no, which has profound impact on how web services should be built. [Give a simple example here.] In essence, this means giving each ``priority'', in terms of resources and scheduling, is cost-inefficient and suboptimal. [Give a simple example. resources wasted for users who are screwed already]
\mypara{Our insight} 
Our answer is {\em no}.  More importantly, even if the requests have no application-level differences (\eg web vs. video), such heterogeneous QoE sensitivity can still result from the differences in non-backend delay (\eg wide-area network routing, client-side software)~\cite{timecard,dqbarge}.
%Two observations contribute to this conclusion: the non-linear relationship between page load time and QoE~\cite{??} and the fact that the WAN/client delay varies among requests~\cite{timecard,dqbarge}, {\em the QoE sensitivity to backend delay varies among requests.}
For instance, the QoE of a web request that has spent 50ms on wide-area networks tends to be more sensitive to 10ms backend delay than a request that has already spent 500ms on the network. 

Realizing the heterogeneity in QoE's sensitivity to backend delay has profound implications for how backend systems should allocate their limited resources. 
By falsely assuming requests are equally sensitive to the backend delay, traditional web backend (Figure~\ref{fig:intro-overview}(a)) can waste precious resources (\eg wasting resources to optimize requests insensitive to the backend delay) and have suboptimal QoE (\eg using inadequate resources on requests critically dependent on the backend delay). 
Instead, we argue that the web backend system should take into account the QoE's sensitivity to the backend delay, and allocate resources in a way that improves overall QoE by favoring the requests whose QoE is more sensitive to backend delay without hurting other requests' QoE or adding any new resources.
In a trace-driven simulation (\S\ref{subsec:example}), we found that such a QoE-driven resource allocation policy can make user engagement 50\% closer to the optimal outcome (zero backend delay) than a QoE-agnostic baseline.
%\jc{bring up some concrete improvement numbers}
%\jc{need to highlight that this is not because application differents}

\begin{figure}[t]
	\centering
	\vspace{-0.5cm}
	\hspace{0.6cm}
	\includegraphics[width=0.75\textwidth]{figs/intro-overview.pdf}
	\vspace{-0.3cm}
	\caption{We propose to re-architect (a) today's web backend which seeks to minimize the backend delays into (b) a QoE-driven web backend which seeks to minimize the impact of backend delay on QoE.}
	\label{fig:intro-overview}
\end{figure}

%\jc{give a figure to contrast optimization of backend in-isolation vs. QoE-aware.}

%- Research goal: This project proposes that the web service backend should be aware of the QoE sensitivity. This effectively changes how one formulates the web service optimization problem.

%This proposal introduces ``QoE sensitivity'' as a new dimension for web backend QoE optimization, 
We propose to develop {\bf QoE-driven web backend} systems (Figure~\ref{fig:intro-overview}(b)). 
The key difference is that traditional backend seeks to minimize the backend delays, but a QoE-driven backend seeks to minimize the impact of backend delays on user-perceived QoE.

Despite its promise, QoE-driven backend faces two key challenges.
First, we need to estimate the real-time QoE information of each web request with sufficient accuracy when the request reaches the backend. 
Second, we need to enable QoE-driven decision-making at various subsystems of a web backend (\eg per-machine resource allocation, scheduling, replica selection).
%Through developing novel algorithms and architectural components, we show that a {\bf QoE-driven web backend} (Figure~\ref{fig:intro-overview}(b)), which is aware of and embraces the differences of QoE sensitivity across requests, can substantially {\em improve the resource/QoE tradeoffs} of web backend; \ie better QoE without using more resources, or saving resources without degrading QoE. 
% Note that being QoE sensitivity does not require expensive infrastructure changes (\eg adding hardware or changing software stack).

\mypara{Research plan}
We divide the proposed research into three tasks.
%We use the following roadmap to thoroughly examine the benefits and challenges of QoE-sensitivity-aware web service backend.

\begin{packeditemize}
\item{\bf Quantifying potential benefits (Task \#1).}
We will do measurement studies to quantify the improvements brought by the QoE-driven web backend in the wild. It will include the opportunities of QoE-driven optimizations in web backend systems, and use dataset collected from large-scale web services to understand the potential in the real-world traffic patterns.

\item{\bf Estimating per-request QoE in real time (Task \#2).}
We will develop new system components, including tracing infrastructure and QoE prediction models, to estimate the impact of backend delay on requests' QoE in real-time. We explore the possibilities of incremental deployment by reusing the existing tracing and telemetry infrastructure in today's web backend systems.

\item{\bf QoE-driven control algorithms (Task \#3).}
We will develop novel QoE-driven control policies for web backend, including resource allocation, scheduling, and replica selection. Our design goals are (1) that the policies should achieve near-optimal QoE with minimal decision-making overhead, and (2) that their implementation should be amenable to existing systems.

%\item{\bf Impact on QoE fairness (Task \#4).}
%Finally, we will explore appropriate definitions of fairness to help strike a desirable balance between QoE-driven optimization and QoE fairness. This would also help us recognize potential threads of other systems/users taking advantage of the QoE-driven policies of the backend.

\end{packeditemize}


%\jc{why these applications?}

%\jc{Common challenges! getting QoE sensitivity, fairness definition!}

\mypara{PI qualification}
The PI's expertise includes computer networking, Internet QoE, and data analytics systems.
He has published 11 peer-reviewed research papers (6 first-authored) in top-tier networking and system conferences (\ie SIGCOMM, NSDI, CoNEXT).
More importantly, the PI has a deep understanding of Internet QoE. His doctoral dissertation, titled ``Enabling Data-Driven Optimization of Quality of Experience in Internet Applications'', is among the first systematic studies to apply data-driven approach to improving Internet  QoE. The dissertation won the CMU SCS Doctoral Dissertation Award and was nominated for ACM Dissertation Award.
During his PhD and postdoctoral years, he has extensive collaborations with Microsoft Research, Conviva Inc., and Google Research. These strong connections will help the proposed research to gain insights from the industry and provide viable paths to deployment.


% \vspace{0.2cm}
% \noindent{\em Thrust \#1: How much potential benefit can we get?}

% \vspace{0.2cm}
% \noindent{\em Thrust \#2: How to re-architect web services to be QoE-aware?}

% \vspace{0.2cm}
% \noindent{\em Thrust \#3: How to propagate user-perceived QoE information?}

%- This project proposes to re-architect the web service backend by making it QoE-aware. Our roadmap has three steps.\\
%1. XXX\\
%2. YYY\\
%3. ZZZ



% \vspace{2cm}
% User-perceived quality of experience (QoE) is one of the driving forces behind the Internet ecosystem, which consists of {\em subsystems}, \eg datacenters, CDNs, cellular carriers, backbone networks, content providers, who share resources across users. 
% % End-to-end Quality of Experience (QoE) is the driving force behind today's Internet application ecosystem, which includes several subsystem
% % The Internet application ecosystem consists of many subsystems, Cloud, ISP, CDNs, etc, and 
% Thus, one fundamental question is {\em how to share resources across users in a way that optimizes their overall QoE?}
% The primary constraint is that these subsystems are {\em federated}: it is impractical to orchestrate a global optimization where they relinquish the control on how their resources are shared. 
% Instead, the conventional wisdom has been that each subsystem shares its resources among users in a way that optimizes the overall performance metrics within its limit and imposes no differentiation between users if they are ``functionally'' identical (\ie same service, business relationship, etc).

% In contrast, we are driven by a simple observation derived directly from the federated nature of the Internet ecosystem.
% In a subsystem, there is {\em a substantial heterogeneity} among its users with respect to how sensitive their QoE is to the performance of the subsystem, even if these users are functionally identical. 
% Thus, the right question to ask is {\em not} how a subsystem should optimize the overall performance among users; instead, it should minimize {\em overall impact on user-perceived QoE}, which requires treating users differently, rather than equally, depending on how much impact it has on the user's perceived QoE.

% In this proposal, we apply this idea to improving QoE of web services.

% \mypara{Research goals}

% \noindent {\bf Intellectual Merit.~~}
% This proposal applies this idea in the context of cloud services. 
% \jc{what it means to cloud services? requests are going to be treated differently, etc} 
% Specifically, this idea can be applied to many services inside a cloud. \jc{talk about more applications}
% In this project, we plan to answer three key question:

% First, how much potential benefit does this idea have?

% Second, how to design a QoE-aware cloud scheduling/resource allocation mechanism?

% Third, how to propogate QoE information from users to the cloud?

% \noindent {\bf Broader Impacts.~~}


% \noindent {\bf Keywords.~~} 



% QoE matters to everyone!

% \subsection{Missed Opportunities}
% \begin{itemize}

% \item Today's tenant: every user should be treated with the same performance goal. Implicit assumption is that the impact of a subsystem is the same on all users.

% \item However, the federated architecture means:\\
% 1. QoE can be affected by any subsystem\\
% 2. Each subsystem serves users with different QoE sensitivities.

% \item Fundamental mismatch: some users who are less sensitive to the subsystem get over-optimized, while others who are more sensitive to the subsystem get under-optimized.

% \item New approach: minimize the overall impact on QoE. 

% \end{itemize}

% \subsection{This proposal: Making Cloud QoE-Aware}
% \begin{itemize}

% \item How the cloud works today -- agnostic to QoE

% \item QoE curve

% \item Examples of how things can be done differently!

% \end{itemize}


% \subsection{Research Roadmap:}
% \begin{itemize}

% \item How much potential benefit does this idea have?

% \item How to design a QoE-aware cloud scheduling/resource allocation mechanism?

% \item How to propagate QoE information from users to the cloud?

% \end{itemize}

\end{comment}
%
%\input{research}
%
%%% \input{example}
%
%%\input{example_junchen}
%
%\input{approach}
%
%\input{plan}
%
%\input{broader}
%
%\input{prior}

\newpage
\pagenumbering{arabic}
\renewcommand{\thepage} {E--\arabic{page}}
%\input{collaboration}


\newpage
\pagenumbering{arabic}
\renewcommand{\thepage} {F--\arabic{page}}
%\input{NSFData}

%%%%%%%%%%%%%%%%%%%%%%%%%%%%%%%%%%%%%%%%
%% F - BIOGRAPHICAL SKETCHES
%% provided separately - see cv_nsf.tex
%\newpage
%\section{Biographical Sketches}
%\newpage
%
%\addtolength{\voffset}{1in}
%\addtolength{\hoffset}{1in}
%
%\includepdf[pages=1-2]{Bios/Chien-NSF-Bio-2016v2}
%
%%%\textbf{John Birge}
%\includepdf[pages=1-2]{Bios/jrb_nsf_Biosketch_2016}
%\newpage
%%%\textbf{Victor Zavala}
%\includepdf[pages=1-2]{Bios/zavala-nsfbio.pdf}
%\addtolength{\voffset}{-1in}
%\addtolength{\hoffset}{-1in}

%%%%%%%%%%%%%%%%%%%%%%%%%%%%%%%%%%%%%%%
%%
%(8.b) Results of Prior Support (2 pages per person). Provide
%information only for the PI(s), each co-PI, and senior personnel, for
%contributions to research and education in science and engineering
%over the past five years (from any funding source). Include a brief
%statement of results of funded projects.


%%%%%%%%%%%%%%%%%%%%%%%%%%%%%%%%%%%%%%%
% E - REFERENCES CITED

\newpage
\pagenumbering{arabic}
\renewcommand{\thepage} {E--\arabic{page}}

\bibliographystyle{abbrv}
\bibliography{Bib/sc16-references,Bib/seec,Bib/zccloud1,Bib/zccloud2,Bib/zccloud4,Bib/opp-stranded-power,Bib/UAP2015,Bib/chien,Bib/xps-andrew,Bib/versioning,Bib/10x10ref_full,Bib/jiang,Bib/jiang-dds-hotcloud,Bib/jiang-chameleon,Bib/jiang-dds-socc,Bib/CERES-March2017,Bib/socc-refs}

%\newpage

\addtolength{\voffset}{1in}
\addtolength{\hoffset}{1in}

%% \subsection{Andrew Chien}
%\includepdf[pages=1-2]{Bios/chien-prior.pdf}

%% \subsection{Hank Hoffmann}

%%\subsection{Junchen Jiang}}

%%%%%%%%%%%%%%%%%%%%%%%%%%%%%%%%%%%%%%%
%%
%% conflicts
\addtolength{\voffset}{-1in}
\addtolength{\hoffset}{-1in}


%%%%%%%%%%%%%%%%%%%%%%%%%%%%%%%%%%%%%%%
% H - CURRENT AND PENDING SUPPORT
% provided separately - see cv_nsf.tex
%% \newpage
%% \pagenumbering{arabic}

%% \renewcommand{\thepage} {F--\arabic{page}}

%% \newpage
%% \includepdf[page=1-2]{Admin/Chien-CP}

%%%%%%%%%%%%%%%%%%%%%%%%%%%%%%%%%%%%%%%
% G - BUDGET JUSTIFICATION

%% \newpage
%% \pagenumbering{arabic}
%% \renewcommand{\thepage} {G--\arabic{page}}
%% ~
%%\noindent{\Large \bf BUDGET JUSTIFICATION}

%% \includepdf[]{Admin/Chien-Budget-Just}

%%%%%%%%%%%%%%%%%%%%%%%%%%%%%%%%%%%%%%%
% H - DATA MANAGEMENT PLAN


%%  \addtolength{\voffset}{1in}
%%  \addtolength{\hoffset}{1in}

%% \newpage
%% \pagenumbering{arabic}
%% \setcounter{page}{1}
%% \renewcommand{\thepage} {H--\arabic{page}}
%% ~

%% \addtolength{\voffset}{-1in}
%% \addtolength{\hoffset}{-1in}

%\newpage
% \include{nsfdataman}
%\newpage

%%%%%%%%%%%%%%%%%%%%%%%%%%%%%%%%%%%%%%%
% I - FACILITIES & RESOURCES

%% \newpage
%% \pagenumbering{arabic}
%% \renewcommand{\thepage} {I--\arabic{page}}
%% %%\noindent{\Large \bf FACILITIES, EQUIPMENT, AND OTHER RESOURCES}\\

%\newpage
% \include{facilities}
%\newpage

%% \addtolength{\voffset}{1in}
%% \addtolength{\hoffset}{1in}

%% \includepdf[page=1-2]{Admin/Chien-Facilities}


%% \addtolength{\voffset}{-1in}
%% \addtolength{\hoffset}{-1in}

\end{document}
