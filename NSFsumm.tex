%!TEX root = main.tex

\vskip -.75em
\noindent {\bf Summary.~~} 
End-to-end QoE is the driving force behind Internet application ecosystem. The Internet application ecosystem consists of many subsystems, Cloud, ISP, CDNs, etc, and the way they work today follows an simple assumption that a subsystem should optimize all users using the same performance objective, if they look functionally the same (same service, business relationship, etc).

However, the Internet application ecosystem is essentially a faderated architecture. It has two profound implications: (1) For each end user, the perceived QoE can be affected by any subsystem along the way. (2) For each subsystem, it serves users whose QoE has different sensitivity to its quality.

This proposal is based on a simple insight: a subsystem should treat each user differently depending on how much its impact on the user's quality is, even if they look the same functionally.
This means that \jc{briefly describe why this is beneficial}

\noindent {\bf Intellectual Merit.~~}
This proposal applies this idea in the context of cloud services. 
\jc{what it means to cloud services? requests are going to be treated differently, etc} 
Specifically, this idea can be applied to many services inside a cloud. \jc{talk about more applications}
In this project, we plan to answer three key question:

First, how much potential benefit does this idea have?

Second, how to design a QoE-aware cloud scheduling/resource allocation mechanism?

Third, how to propogate QoE information from users to the cloud?

\noindent {\bf Broader Impacts.~~}


\noindent {\bf Keywords.~~} 

