\vskip -.75em
\mypara{Summary}
User-perceived quality of experience (QoE) is a key driving force behind the Internet ecosystem, yet many users still suffer from suboptimal QoE. 
Unlike previous approaches (\eg cutting tail backend delay, pushing caches closer to end users), this proposal introduces a new dimension to QoE optimization: {\bf awareness of QoE sensitivity}, \ie how sensitive a user's QoE is to the delay of the web service.
For instance, a web request that has spent 50ms on wide-area networks before reaching the web server is more likely to be affected by 10ms delay of the web service than a request that has already spent 500ms on the network. 
Such difference of QoE sensitivity is largely ignored in today's web services, but it has profound implications, especially when the requests compete for limited shared resources.
Our early results show that by making existing web service aware of QoE sensitivity, we could significantly improve web service QoE and save resources compared to designs that are agnostic to QoE sensitivity.
In this proposal, we plan to thoroughly investigate the idea of making web services aware of QoE-sensitivity in three steps.
(1) We quantify its potential benefits in QoE and resource savings.
(2) We propose novel algorithms for QoE-aware scheduling and resource allocation of web services. 
(3) We present novel system designs and implement prototypes that make web services QoE-aware in practice.


% Thus, the goal of each subsystems in a large web service, such as web server or key-value store, should be to optimize the overall QoE of many users given limited resources. 
% A common approach to achieving this goal is for each subsystem to optimize some ``local'' performance metrics measured within its scope (\eg server-side delay) over all users, and the intuition is that if each subsystem follows the approach, it will optimize the overall QoE of users. 
% We argue, however, that this approach only achieves suboptimal QoE and can use more resources than necessary. 
% Our key observation is that {\em the impact of a subsystem's performance on a user's perceived QoE varies greatly among users} (modulo web page type, business relationship), so when sharing resources across users, each subsystem should take into account its impact on each user's QoE.
% One typical sources 
% This has profound implication on how web services should be optimized, and opens up many several new opportunities.


\section{Overview}

User-perceived quality of experience (QoE) is one of the driving forces behind the Internet ecosystem, which consists of {\em subsystems}, \eg datacenters, CDNs, cellular carriers, backbone networks, content providers, who share resources across users. 
% End-to-end Quality of Experience (QoE) is the driving force behind today's Internet application ecosystem, which includes several subsystem
% The Internet application ecosystem consists of many subsystems, Cloud, ISP, CDNs, etc, and 
Thus, one fundamental question is {\em how to share resources across users in a way that optimizes their overall QoE?}
The primary constraint is that these subsystems are {\em federated}: it is impractical to orchestrate a global optimization where they relinquish the control on how their resources are shared. 
Instead, the conventional wisdom has been that each subsystem shares its resources among users in a way that optimizes the overall performance metrics within its limit and imposes no differentiation between users if they are ``functionally'' identical (\ie same service, business relationship, etc).

In contrast, we are driven by a simple observation derived directly from the federated nature of the Internet ecosystem.
In a subsystem, there is {\em a substantial heterogeneity} among its users with respect to how sensitive their QoE is to the performance of the subsystem, even if these users are functionally identical. 
Thus, the right question to ask is {\em not} how a subsystem should optimize the overall performance among users; instead, it should minimize {\em overall impact on user-perceived QoE}, which requires treating users differently, rather than equally, depending on how much impact it has on the user's perceived QoE.

In this proposal, we apply this idea to improving QoE of web services.

\mypara{Research goals}

\noindent {\bf Intellectual Merit.~~}
This proposal applies this idea in the context of cloud services. 
\jc{what it means to cloud services? requests are going to be treated differently, etc} 
Specifically, this idea can be applied to many services inside a cloud. \jc{talk about more applications}
In this project, we plan to answer three key question:

First, how much potential benefit does this idea have?

Second, how to design a QoE-aware cloud scheduling/resource allocation mechanism?

Third, how to propogate QoE information from users to the cloud?

\noindent {\bf Broader Impacts.~~}


\noindent {\bf Keywords.~~} 



QoE matters to everyone!

\subsection{Missed Opportunities}
\begin{itemize}

\item Today's tenant: every user should be treated with the same performance goal. Implicit assumption is that the impact of a subsystem is the same on all users.

\item However, the federated architecture means:\\
1. QoE can be affected by any subsystem\\
2. Each subsystem serves users with different QoE sensitivities.

\item Fundamental mismatch: some users who are less sensitive to the subsystem get over-optimized, while others who are more sensitive to the subsystem get under-optimized.

\item New approach: minimize the overall impact on QoE. 

\end{itemize}

\subsection{This proposal: Making Cloud QoE-Aware}
\begin{itemize}

\item How the cloud works today -- agnostic to QoE

\item QoE curve

\item Examples of how things can be done differently!

\end{itemize}


\subsection{Research Roadmap:}
\begin{itemize}

\item How much potential benefit does this idea have?

\item How to design a QoE-aware cloud scheduling/resource allocation mechanism?

\item How to propagate QoE information from users to the cloud?

\end{itemize}