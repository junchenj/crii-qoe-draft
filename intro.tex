\vskip -.75em
\jc{yet another QoE improvement doesn't sound very exciting}
\mypara{Summary}
User-perceived quality of experience (QoE) is a key driving force behind the Internet ecosystem, including web services. 
Despite much effort to optimize web services, many users still suffer from suboptimal QoE, such as excessively long page load time.
Unlike previous approaches such as cutting tails of response time or pushing caches closer to end users, this proposal introduces a new dimension for QoE optimization: the {\bf sensitivity of QoE to backend delay}, \ie how sensitive a user's QoE is to the web service backend delay.
For instance, a web request that has spent 50ms on wide-area networks before reaching the web server is much more likely to be affected by 10ms delay of the web service than a request that has already spent 500ms on the network. 
Such discrepancy in QoE sensitivity is largely ignored in today's web services, but it has profound implications, especially when many requests compete the limited resources.
Our early results show that by making existing the web service backend aware of QoE sensitivity, we could improve both QoE and resource efficiency than existing systems that are agnostic to QoE sensitivity.
In this proposal, we show various benefits of making web services aware of QoE sensitivity (\eg improving request scheduling and resource allocation) and provide a technical roadmap to address common challenges  (\eg estimating QoE sensitivity).
% (1) We quantify its potential benefits in QoE and resource savings.
% (2) We propose novel algorithms for QoE-aware scheduling and resource allocation of web services. 
% (3) We present novel system designs and implement prototypes that make web services QoE-aware in practice.


% Thus, the goal of each subsystems in a large web service, such as web server or key-value store, should be to optimize the overall QoE of many users given limited resources. 
% A common approach to achieving this goal is for each subsystem to optimize some ``local'' performance metrics measured within its scope (\eg server-side delay) over all users, and the intuition is that if each subsystem follows the approach, it will optimize the overall QoE of users. 
% We argue, however, that this approach only achieves suboptimal QoE and can use more resources than necessary. 
% Our key observation is that {\em the impact of a subsystem's performance on a user's perceived QoE varies greatly among users} (modulo web page type, business relationship), so when sharing resources across users, each subsystem should take into account its impact on each user's QoE.
% One typical sources 
% This has profound implication on how web services should be optimized, and opens up many several new opportunities.


\section{Overview}

% - QoE is important and our goal is to improve QoE for Web Services.
Large-scale web services (such as Microsoft, Facebook, Google) depend critically on maintaining high {\em Quality of Experience} (QoE) perceived by millions of users. 
Thus, a fundamental question is how to share resources of web service backend among users so as to optimize their QoE. 
%Their business models, based on advertisement or subscription, are driven by user engagement, for which QoE is believed to play a vital role (among other factors such as content, user interfaces).

\mypara{Limitation of today's web services}
One of the determining factors of QoE is the page load time, which generally consists of three types of latency: wide-area network (WANs) delay, client-side delay, and backend delay.
%- Web services, like applications running in the cloud, have been basing their optimizations on the goal of improving server-side latency (sometimes the fraction of users meeting some fixed deadline)
Because of the federated nature of Internet architecture, web service providers do not have full control over all types of delays.
%---to them, WANs and clients devices are largely blackboxes operated, not by the web services, but by ISPs, cellular carriers, and device vendors.
%(while web browsers and apps are developed by the web service providers, the client-side performance is largely decided by how OS share resources among multiple applications).
Thus, instead of optimizing for QoE directly, today's web services focus on reducing the {\em backend delay} with the assumption that different requests have the same {\em QoE sensitivity to backend delay}\footnote{Modulo the content-specific (\eg web page type) or user-specific (\eg free vs. premium subscription) factors.}.
That is, a backend delay of $n$~ms has the same effect on the QoE of any request.
As a result, they strive to minimize the mean/tail values of the backend delay or the fraction of requests whose backend delays exceed a predetermined deadline (\eg 300ms).
%- This project takes a step back and asks a different question: does the latency have the same impact on user QoE? 
% In doing so, 
%all requests are optimized with the same objective function of backend response time; 
% an implicit assumption is that different requests have the same {\em QoE sensitivity to backend delay} (modulo content-/user-specific factors, such as web page type or subscription type, etc);
% that is, a backend delay of $n$~ms has the same effect on the QoE of different requests.

However, we take a step back and ask: is it true that every request has the {\em same} QoE sensitivity to backend delay?

%- The answer is no, which has profound impact on how web services should be built. [Give a simple example here.] In essence, this means giving each ``priority'', in terms of resources and scheduling, is cost-inefficient and suboptimal. [Give a simple example. resources wasted for users who are screwed already]
The answer is {\em no}, which, as we will see, has profound implications for improving the way the web service backend shares its resources among users. 
Because of the non-linear relationship between page load time and QoE~\cite{??} and the fact that the WAN/client delay varies among requests~\cite{timecard,dqbarge}, {\em the QoE sensitivity to backend delay varies among requests.}
For instance, a web request that has spent 50ms on wide-area networks before reaching the web server is more likely to be affected by 10ms delay of the web service than a request that has already spent 500ms on the network. 
Given the heterogeneity of QoE sensitivity among requests, it is clear that the assumption of backend delay having the same impact to QoE of any request can lead to low cost efficiency (wasting resources to optimize requests insensitive to the backend delay), as well as suboptimal QoE (spending inadequate resources on requests critically dependent on the backend delay). 
\jc{bring up some concrete improvement numbers}
\jc{need to highlight that this is not because application differents}

\jc{give a figure to contrast optimization of backend in-isolation vs. QoE-aware.}

%- Research goal: This project proposes that the web service backend should be aware of the QoE sensitivity. This effectively changes how one formulates the web service optimization problem.
\mypara{Research goals}
This proposal introduces ``QoE sensitivity'' as a new dimension to improving the web service backend. 
We show that by making the backend aware of QoE sensitivity, one can substantially {\em improve resource/QoE tradeoffs} for web service backend---\ie better QoE without using more resources, or saving resources without degrading QoE. 
% Note that being QoE sensitivity does not require expensive infrastructure changes (\eg adding hardware or changing software stack).

We divide the proposed research into four thrusts.
We use the following roadmap to thoroughly examine the benefits and challenges of QoE-sensitivity-aware web service backend.


\vspace{0.2cm}
\noindent{\em Thrust \#1: Making job scheduling aware of QoE sensitivity}

\vspace{0.2cm}
\noindent{\em Thrust \#2: Making resource allocation aware of QoE sensitivity}

\vspace{0.2cm}
\noindent{\em Thrust \#3: Making congestion control aware of QoE sensitivity}

\jc{why these applications?}

\jc{Common challenges! getting QoE sensitivity, fairness definition!}

\mypara{PI qualifications}


% \vspace{0.2cm}
% \noindent{\em Thrust \#1: How much potential benefit can we get?}

% \vspace{0.2cm}
% \noindent{\em Thrust \#2: How to re-architect web services to be QoE-aware?}

% \vspace{0.2cm}
% \noindent{\em Thrust \#3: How to propagate user-perceived QoE information?}

%- This project proposes to re-architect the web service backend by making it QoE-aware. Our roadmap has three steps.\\
%1. XXX\\
%2. YYY\\
%3. ZZZ



% \vspace{2cm}
% User-perceived quality of experience (QoE) is one of the driving forces behind the Internet ecosystem, which consists of {\em subsystems}, \eg datacenters, CDNs, cellular carriers, backbone networks, content providers, who share resources across users. 
% % End-to-end Quality of Experience (QoE) is the driving force behind today's Internet application ecosystem, which includes several subsystem
% % The Internet application ecosystem consists of many subsystems, Cloud, ISP, CDNs, etc, and 
% Thus, one fundamental question is {\em how to share resources across users in a way that optimizes their overall QoE?}
% The primary constraint is that these subsystems are {\em federated}: it is impractical to orchestrate a global optimization where they relinquish the control on how their resources are shared. 
% Instead, the conventional wisdom has been that each subsystem shares its resources among users in a way that optimizes the overall performance metrics within its limit and imposes no differentiation between users if they are ``functionally'' identical (\ie same service, business relationship, etc).

% In contrast, we are driven by a simple observation derived directly from the federated nature of the Internet ecosystem.
% In a subsystem, there is {\em a substantial heterogeneity} among its users with respect to how sensitive their QoE is to the performance of the subsystem, even if these users are functionally identical. 
% Thus, the right question to ask is {\em not} how a subsystem should optimize the overall performance among users; instead, it should minimize {\em overall impact on user-perceived QoE}, which requires treating users differently, rather than equally, depending on how much impact it has on the user's perceived QoE.

% In this proposal, we apply this idea to improving QoE of web services.

% \mypara{Research goals}

% \noindent {\bf Intellectual Merit.~~}
% This proposal applies this idea in the context of cloud services. 
% \jc{what it means to cloud services? requests are going to be treated differently, etc} 
% Specifically, this idea can be applied to many services inside a cloud. \jc{talk about more applications}
% In this project, we plan to answer three key question:

% First, how much potential benefit does this idea have?

% Second, how to design a QoE-aware cloud scheduling/resource allocation mechanism?

% Third, how to propogate QoE information from users to the cloud?

% \noindent {\bf Broader Impacts.~~}


% \noindent {\bf Keywords.~~} 



% QoE matters to everyone!

% \subsection{Missed Opportunities}
% \begin{itemize}

% \item Today's tenant: every user should be treated with the same performance goal. Implicit assumption is that the impact of a subsystem is the same on all users.

% \item However, the federated architecture means:\\
% 1. QoE can be affected by any subsystem\\
% 2. Each subsystem serves users with different QoE sensitivities.

% \item Fundamental mismatch: some users who are less sensitive to the subsystem get over-optimized, while others who are more sensitive to the subsystem get under-optimized.

% \item New approach: minimize the overall impact on QoE. 

% \end{itemize}

% \subsection{This proposal: Making Cloud QoE-Aware}
% \begin{itemize}

% \item How the cloud works today -- agnostic to QoE

% \item QoE curve

% \item Examples of how things can be done differently!

% \end{itemize}


% \subsection{Research Roadmap:}
% \begin{itemize}

% \item How much potential benefit does this idea have?

% \item How to design a QoE-aware cloud scheduling/resource allocation mechanism?

% \item How to propagate QoE information from users to the cloud?

% \end{itemize}
